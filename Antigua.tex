\documentclass{article}
\usepackage[chorded]{songs}
\usepackage{biblatex}
\usepackage[landscape]{geometry}

%\noversenumbers

\begin{document}

\begin{titlepage}
   \vspace*{\stretch{1.0}}
   \begin{center}
      \Huge\textbf{The Antigua songbook}\\
      \Large\textit{Vaga, Sylvestre, Cruising Bird}
   \end{center}
   \vspace*{\stretch{2.0}}
\end{titlepage}
\songsection{Chanson française}

\begin{songs}{}
\linespread{0.1}

\beginsong{Il est libre Max}[by={Hervé Christiani},sr={Lolo solo}]

\beginverse
\gtab{Em}{022000} \gtab{C}{032010}\gtab{D}{000232}
\[Em]Il met de la magie mine de rien \[C]dans tout c'qu'il fait
Il \[D]a le sourire facile même pour \[Em]les imbéciles
Il \[Em]s'amuse bien il tombe \[C]jamais dans les pièges
\[D]Il s'laisse pas étourdir par \[Em]les néons des manêges
Il \[Em]vit sa vie sans s'occuper \[C]des grimaces
Que font \[D]autour de lui les poissons\[Em] dans la nasse
\endverse

\beginchorus
\[Em]Il est libre Max, \[C]il est libre Max
\[D]Y'en a même qui disent qu'ils l'ont \[Em]vu voler
\endchorus

\beginverse
Il travaille un p'tit peu quand son orps est d'accord
Pour lui faut pas s'en faire il sait doser son effort
Dans l'panier d'crabes il joue pas les homards
Il cherche pas à tout prix à faire des bulles dans la mare
\endverse

\beginverse
Il r'garde autour de lui avec les yeux de l'amour
Avant qu't'aies rien pu dire il t'aime déjà au départ
Il fait pas d'bruit il joue pas du tambour
Mais la statue de marbre lui sourit dans la cour
\endverse

\beginverse
Et bien sûr toutes les filles lui font leurs yeux de velours
Lui pour leur faire plaisir il leur raconte des histoires
Il les emmène par delà les labours
Chevaucher les licornes à la tombée du soir
\endverse

\beginverse
Comme il n'a pas d'argent pour faire le grand voyageur
Il va parler souvent aux habitants de son coeur
Qu'est-ce qui s'racontent c'est ça qu'il faudrait savoir
Pour avoir comme lui autant d'amour dans l'regard
\endverse

\endsong

\beginsong{La cabane du pêcheur}[by={Francis Cabrel},sr={demandé par Sophie}]
\beginverse
\[E]Le soir \[B7/4]tombait de tout son \[E]poids \[B7/4]
\[E]Au dess\[B7/4]us de la ri\[E]vière \[B7/4]
\[C#m]Je rangeais \[B]mes cannes
On ne voyait plus \[A]que  \[E]du \[B7/4]feu
\[E]Je l'ai vu \[B7/4]s'approcher
\[E]La tête aille\[B7/4]urs dans ses \[E]prières \[B7/4]
\[C#m]Il m'a semblé \[B]voir trop briller \[A]ses yeux

Huuumm  -- Je lui  \[B7/4] ai dit     \[E]     \[B7/4]
E Si tu p \[B7/4] leures pour\[E] un garçon \[B7/4]
E Tu seras \[B7/4]  pas la der\[E]nière \[B7/4]
\[C#m] Souvent les poi B ssons sont bien pl A  us affectueux
E Va faire un petit t \[B7/4] our, respir\[E]e le grand a \[B7/4]  ir
\[C#m] Après je te parlerai de l'am B  our
Si je me souviens un peu  A
\endverse

\beginchorus
ouhhh\[E]Elle m' \[B7/4] a dit    \[E]      \[B7/4]
\[F#m] Elle m'a dit justement c'est ce que je voudra B  is savoir
Et j'ai dit " A  viens t'asseoir dans la \[C#m]  cabane du pécheur  D
C'est un mauva A is rêve, oublie-l\[E] e !
Tes rêves s \[F#m] ont toujours trop clair B  s ou trop noirs
Alors  A viens faire toi-même le mélange des couleurs   \[C#m]
Sur les murs de la cabane du pécheur   A
Viens t'asseoir " \[E]                \[B7/4]
Je lui ai dit
Le monde est pourtant pas si loin
On voit les lumières
Et la terre peut faire
Tous les bruits qu'elle veut
Y'a sûrement quelqu'un qui écoute
La-haut dans l'univers
Peut-être tu demandes plus qu'il ne peut ?
\endchorus
\endsong

\beginsong{Femme libérée}[by={Christian Dingler},sr={demandé par Bibi}]

\beginverse
Elle est abonnée à Marie-Claire
Dans l'Nouvel Ob's elle ne lit que Bretecher
Le Monde y'a longtemps qu'elle fait plus semblant
Elle achète Match en cachette c'est bien plus marrant
\endverse

\beginchorus
Ne la laisse pas tomber
Elle est si fragile
Etre une femme libérée tu sais c'est pas si facile
Ne la laisse pas tomber
Elle est si fragile
Etre une femme libérée tu sais c'est pas si facile
\endchorus

\beginverse
Au fond de son lit un macho s'endort
Qui ne l'aimera pas plus loin que l'aurore
Mais elle s'en fout elle s'éclate quand même
Et lui ronronne des tonnes de "Je t'aime"
\endverse

\beginverse
Sa première ride lui fait du souci
Le reflet du miroir pèse sur sa vie
Elle rentre son ventre à chaque fois qu'elle sort
Même dans "Elle" ils disent qu'il faut faire un effort
\endverse

\beginverse
Elle fume beaucoup elle a des avis sur tout
Elle aime raconter qu'elle sait changer une roue
Elle avoue son âge celui de ses enfants
Et goûte même un p'tit joint de temps en temps
\endverse

\endsong

\beginsong{L'autre Finistère}[by={Les Innocents},sr={demandé par Sophie}]

\beginverse
Com-pren-drais-tu ma belle qu'un jour
fa-ti-gué j'ail-le me bri-ser la voix
une der-nière fois à cent vingt dé-ci-bels
contre un grand châ-tai-gner
d'a-mour pour toi
Trouverais-tu cruel
Que le doigt sur la bouche
Je t'emmène, hors des villes
En un fort, une presqu'île
Oublier nos duels
Nos escarmouches
Nos peurs imbéciles
On irait y attendre
La fin des combats
Jeter aux vers, aux vautours
Tous nos plus beaux discours
Ces mots qu'on rêvait d'entendre
Et qui n'existent pas
Y devenir sourds
\endverse

\beginchorus
Il est un es-tuaire à nos fleu-ves de sou-pirs
où l'eau mê-le nos mys-tères et nos bel-les dif-fé-rences
j'y ap-pren-drai à me taire et tes lar-mes re-te-nir
dans cet au-tre Fi-nis-tère aux lon-gues pla-ges de si-len ce
\endchorus

\beginverse
Trouverais-tu cruel
Que le doigt sur la bouche
Je t'emmène, hors des villes
En un fort, une presqu'île
Oublier nos duels
Nos escarmouches
Nos peurs imbéciles
On irait y attendre
La fin des combats
Jeter aux vers, aux vautours
Tous nos plus beaux discours
Ces mots qu'on rêvait d'entendre
Et qui n'existent pas
Y devenir sourds
\endverse

\beginverse
Bien sûr on se figure
Que le monde est mal fait
Que les jours nous abiment
Comme de la toile de Nîmes
Qu'entre nous, il y a des murs
Qui jamais ne fissurent
Que même l'air nous opprime
Et puis on s'imagine
Des choses et des choses
Que nos liens c'est l'argile
Des promesses faciles
Sans voir que sous la patine
Du temps, il y a des roses
Des jardins fertiles
\endverse

\beginverse
Car là-haut dans le ciel
Si un jour je m'en vais
Ce que je voudrais de nous
Emporter avant tout
C'est le sucre, et le miel
Et le peu que l'on sait
N'être qu'à nous
\endverse
\endsong

\beginsong{Elle a les yeux revolvers}[by={Marc Lavoine},sr={demandé par Vivi qui le trouve trop sexy}]

\beginverse
Un peu spéciale, elle est célibataire
Le visage pâle, les cheveux en arrière
Et j'aime ça
Elle se dessine sous des jupes fendues
Et je devine des histoires défendues
C'est comme ça
Tellement si belle quand elle sort
Tellement si belle, je l'aime tellement si fort
\endverse

\beginchorus
Elle a les yeux revolver, elle a le regard qui tue
Elle a tiré la première, m'a touché, c'est foutu
Elle a les yeux revolver, elle a le regard qui tue
Elle a tiré la première, elle m'a touché, c'est foutu
\endchorus

\beginverse
Un peu larguée, un peu seule sur la terre
Les mains tendues, les cheveux en arrière
Et j'aime ça
A faire l'amour sur des malentendus
On vit toujours des moments défendus
C'est comme ça
Tellement si femme quand elle mord
Tellement si femme, je l'aime tellement si fort
\endverse

\beginverse
Son corps s'achève sous des draps inconnus
Et moi je rêve de gestes défendus
C'est comme ça
Un peu spéciale, elle est célibataire
Le visage pâle, les cheveux en arrière
Et j'aime ça
Tellement si femme quand elle dort
Tellement si belle, je l'aime tellement si fort
\endverse

\endsong

\beginsong{Les Champs Elysées}[by={Joe Dassin},sr={spéciale dédicace à Oscar né à Paris XII}]

\beginverse
Je m'baladais sur l'avenue le cœur ouvert à l'inconnu
J'avais envie de dire bonjour à n'importe qui
N'importe qui et ce fut toi, je t'ai dit n'importe quoi
Il suffisait de te parler, pour t'apprivoiser
\endverse

\beginchorus
Aux Champs-Elysées, aux Champs-Elysées
Au soleil, sous la pluie, à midi ou à minuit
Il y a tout ce que vous voulez aux Champs-Elysées
\endchorus

\beginverse
Tu m'as dit "J'ai rendez-vous dans un sous-sol avec des fous
Qui vivent la guitare à la main, du soir au matin"
Alors je t'ai accompagnée, on a chanté, on a dansé
Et l'on n'a même pas pensé à s'embrasser
\endverse

\beginverse
Hier soir deux inconnus et ce matin sur l'avenue
Deux amoureux tout étourdis par la longue nuit
Et de l'Étoile à la Concorde, un orchestre à mille cordes
Tous les oiseaux du point du jour chantent l'amour
\endverse
\endsong

\beginsong{Africa}[by={Rose Laurens},sr={Pour Vivi la reine du dancefloor}]

\beginverse
Je suis amoureuse d'une terre sauvage
Un sorcier vaudou m'a peint le visage
Son gris-gris me suit au son des tam-tams
Parfum de magie sur ma peau blanche de femme
\endverse

\beginchorus
Africa
J'ai envie de danser comme toi
De m'offrir à ta loi
Africa
De bouger à me faire mal de toi
Et d'obéir à ta voix
Africa
\endchorus

\beginverse
Je danse pied nus sous un soleil rouge
Les dieux à genoux ont le cœur qui bouge
Le feu de mon corps devient un rebelle
Le cri des gourous a déchiré le ciel
\endverse

\beginverse
Dangereuse et sensuelle, sous ta pluie sucrée
Panthère ou gazelle je me suis couchée
Au creux de tes griffes je suis revenue
A l'ombre des cases je ferai ma tribu
\endverse
\endsong

\beginsong{Ne partons pas fâchés}[by={Raphaël},sr={Et c'est pour Looooooorent}]

\beginverse
Bien sur qu'on a perdu la guerre, bien sur que je le reconnais
bien sur la vie nous mets le compte, bien la vie c'est une enclume
bien sur que j'aimerais bien te montrer qu'ailleurs on ferait pas que fuir
et bien sur j'ai pas les moyens et quand les poches sont vides alors allons rire
\endverse

\beginchorus
Ne partons pas fâchés, ça n'en vaut pas la peine
\endchorus

\beginverse
Bien sur que les montagnes sont belles, bien sur qu'il y a des vallées
Et les enfants sautent sur les mines, bien sur dans une autre vallée
Bien sur que les poissons ont froids à se traîner la dans la mer
Bien sur que j'ai encore en moi comme un veau avalé de travers
\endverse

\beginverse
Bien sur j'ai la ville dans le ventre, bien sur j'ai vendu ma moto
bien sur je te trouve très jolie, j'ai vraiment envie de te sauter
bien sur la vie nous fait offense bien sur la vie nous fait misère
on ira aussi vite que le vent, même si on a bien souvent ramper
\endverse


\beginverse
Bien sur que je te trouve très belle, bien sur je t'emmènerai à la mer
y'a rien d'autre a faire qu'à se saouler, attendre le jugement dernier
Transplanter la haut dans le Ciel, y parait que c'est pas pareil !
y parait que la vie n'es jamais aussi belle que dans tes rêves que dans tes rêves
\endverse

\beginverse
Et si l'on ne fait rien,
Ne partons pas fâchés, ça n'en vaut pas la peine
Y parait que les petits moineaux...
\endverse
\endsong

\beginsong{Belle Île en Mer}[by={Laurent Voulzy},sr={Pour les langoustes}]

\beginchorus
Belle-Ile-en-Mer
Marie-Galante
Saint-Vincent
Loin Singapour
Seymour Ceylan
Vous c'est l'eau c'est l'eau
Qui vous sépare
Et vous laisse à part
\endchorus

\beginverse
Moi des souvenirs d'enfance
En France
Violence
Manque d'indulgence
Par les différences que j'ai
Café
Léger
Au lait mélangé
Séparé petit enfant
Tout comme vous
Je connais ce sentiment
De solitude et d'isolement
\endverse

\beginverse
Comme laissé tout seul en mer
Corsaire
Sur terre
Un peu solitaire
L'amour je 1' voyais passer
Ohé Ohé
Je l' voyais passer
Séparé petit enfant
Tout comme vous
Je connais ce sentiment
De solitude et d'isolement
\endverse

\beginchorus
Karudea
Calédonie
Ouessant
Vierges des mers
Toutes seules
Tout 1' temps
Vous c'est l'eau c'est l'eau
Qui vous sépare
Et vous laisse à part
Oh oh...
\endchorus

\endsong

\beginsong{La Vie par procuration}[by={JJG},sr={Pour les filles}]

\beginverse
Elle met du vieux pain sur son balcon
Pour attirer les moineaux, les pigeons
Elle vit sa vie par procuration
Devant son poste de télévision
\endverse

\beginverse
Levée sans réveil
Avec le soleil
Sans bruit, sans angoisse
La journée se passe
Repasser, poussière
Y'a toujours à faire
Repas solitaires
En points de repère
\endverse

\beginverse
La maison si nette
Qu'elle en est supecte
Comme tous ces endroits
Où l'on ne vit pas
Les êtres ont cédé
Perdu la bagarre
Les choses ont gagné
C'est leur territoire
\endverse

\beginverse
Le temps qui nous casse
Ne la change pas
Les vivants se fanent
Mais les ombres, pas
Tout va, tout fonctionne
Sans but, sans pourquoi
D'hiver en automne
Ni fièvre, ni froid
\endverse

\beginverse
Elle met du vieux pain sur son balcon
Pour attirer les moineaux, les pigeons
Elle vit sa vie par procuration
Devant son poste de télévision
\endverse

\beginverse
Elle apprend dans la presse à scandale
La vie des autres qui s'étale
Mais finalement, de moins pire en banal
Elle finira par trouver ça normal
\endverse

\beginverse
Elle met du vieux pain sur son balcon
Pour attirer les moineaux, les pigeons
\endverse

\beginverse
Des crèmes et des bains
Qui font la peau douce
Mais ça fait bien loin
Que personne ne la touche
Des mois, des années
Sans personne à aimer
Et jour après jour
L'oubli de l'amour
\endverse

\beginverse
Ses rêves et désirs
Si sages et possibles
Sans cri, sans délire
Sans inadmissible
Sur dix ou vingt pages
De photos banales
Bilan sans mystère
D'années sans lumière
\endverse

\beginverse
Elle met du vieux pain sur son balcon
\endverse
\endsong


\beginsong{Papaoutai}[by={Stromae},sr={Allez Damien}]

\beginverse
Dites-moi d'où il vient
Enfin je saurais où je vais
Maman dit que lorsqu'on cherche bien
On finit toujours par trouver
Elle dit qu'il n'est jamais très loin
Qu'il part très souvent travailler
Maman dit "travailler c'est bien"
Bien mieux qu'être mal accompagné
Pas vrai ?
Où est ton papa ?
Dis-moi où est ton papa ?
Sans même devoir lui parler
Il sait ce qui ne va pas
Ah sacré papa
Dis-moi où es-tu caché ?
Ça doit, faire au moins mille fois que j'ai
Compté mes doigts
\endverse

\beginchorus
Où t'es, papa où t'es ?
Où t'es, papa où t'es ?
Où t'es, papa où t'es ?
Où, t'es où, t'es où, papa où t'es ?
\endchorus

\beginverse
Quoi, qu'on y croit ou pas
Y aura bien un jour où on y croira plus
Un jour ou l'autre on sera tous papa
Et d'un jour à l'autre on aura disparu
Serons-nous détestables ?
Serons-nous admirables ?
Des géniteurs ou des génies ?
Dites-nous qui donne naissance aux irresponsables ?
Ah dites-nous qui, tient,
Tout le monde sait comment on fait les bébés
Mais personne sait comment on fait des papas
Monsieur Je-sais-tout en aurait hérité, c'est ça
Faut l'sucer d'son pouce ou quoi ?
Dites-nous où c'est caché, ça doit
Faire au moins mille fois qu'on a, bouffé nos doigts
\endverse

\beginverse
Où est ton papa ?
Dis-moi où est ton papa ?
Sans même devoir lui parler
Il sait ce qui ne va pas
Ah sacré papa
Dis-moi où es-tu caché ?
Ça doit, faire au moins mille fois que j'ai
Compté mes doigts
\endverse
\endsong

\beginsong{Ma Rue}[by={Zebda},sr={La jeunesse de Nico}]

\beginverse
Dans cette rue y avait
Des espagnols qui n'osaient pas
Montrer qu'ils étaient de vieux réfugiés
Qu'avaient fui les cons et les rois
Dans cette rue y avait
Des français qu'avaient pas de chance ils ont écrit "Vive la France"
Au fronton de leur maison
Dans cette rue y avait
Des portugais fiers, Comme les geôliers de la misère
Quelques arbres fruitiers
Et la pudeur de la terre
\endverse

\beginchorus
C'était
Ma rue, ma famille
Les mamans qui s'égosillent
C'était : va jouer aux billes
C'était ma rue
C'était pas Manille
Non c'était pas les Antilles
Le marteau ou la faucille
C'était ma rue
Les glaces à la vanille
Et les petites qui frétillent
Qui n'étaient pas si gentilles
C'était ma rue
Bonjour les anguilles
Les condés qui nous quadrillent
Moi c'était pas ma Bastille
C'était ma rue
\endchorus

\beginverse
Dans cette rue y avait
L'Afrique et son mea-culpa
D'avoir un autre dieu je crois
Y z'ont trouvé des cons et des croix
Dans cette rue y avait
Tous les ouvriers de la terre
Y z'ont construit des pieds à terre
Qu'ils n'habiteront jamais
Dans cette rue y avait
Des caravanes comme
Des chariots de la colère
Qu'ont pas peur de l'hiver
De la fureur de la terre
\endverse

\beginverse
Dans cette rue je crois
Les enfants n'étaient pas de glace
Quand passait le camion de glace
On tirait des langues étrangères
On était dans les bois
On avait des arcs et des flèches
Quand d'autres avaient des cannes à pêche
Mais l'école elle en veut pas
Un jour on s'est fâchés
On a tout brûlé on a pas eu peur de l'enfer
Quand on s'est réveillés
Derrière des barreaux en fer
\endverse
\endsong

\beginsong{Déjeuner en paix}[by={Stephan Eicher},sr={So gooood}]

\beginverse
J'abandonne sur une chaise le journal du matin
Les nouvelles sont mauvaises d'où qu'elles viennent
J'attends qu'elle se réveille et qu'elle se lève enfin
Je souffle sur les braises pour qu'elles prennent
\endverse

\beginverse
Cette fois je ne lui annoncerai pas
La dernière hécatombe
Je garderai pour moi ce que m'inspire le monde
Elle m'a dit qu'elle voulait si je le permettais
Déjeuner en paix, déjeuner en paix
\endverse

\beginverse
Je vais à la fenêtre et le ciel ce matin
N'est ni rose ni honnête pour la peine
" Est-ce que tout va si mal ? Est-ce que rien ne va bien ?
L'homme est un animal " me dit-elle
\endverse

\beginverse
Elle prend son café en riant
Elle me regarde à peine
Plus rien ne la surprend sur la nature humaine
C'est pourquoi elle voudrait enfin si je le permets
Déjeuner en paix, déjeuner en paix
\endverse

\beginverse
Je regarde sur la chaise le journal du matin
Les nouvelles sont mauvaises d'où qu'elles viennent
" Crois-tu qu'il va neiger ? " me demande-t-elle soudain
" Me feras-tu un bébé pour Noël ? "
\endverse

\beginverse
Et elle prend son café en riant
Elle me regarde à peine
Plus rien ne la surprend sur la nature humaine
C'est pourquoi elle voudrait enfin si je le permets
Déjeuner en paix, déjeuner en paix
\endverse
\endsong

\beginsong{Je danse le mia}[by={I am},sr={Yo yo yo yo}]
\endsong

\beginsong{Dès que le vent soufflera}[by={Renaud},sr={We did it}]

\beginverse
C'est pas l'homme qui prend la mer
C'est la mer qui prend l'homme, tatatin
Moi, la mer, elle m'a pris
Je m'souviens un mardi
J'ai troqué mes santiags
Et mon cuir un peu zone
Contre une paire de docksides
Et un vieux ciré jaune
J'ai déserté les crasses
Qui me disaient "Sois prudent"
La mer, c'est dégueulasse
Les poissons baisent dedans
\endverse

\beginchorus
Dès que le vent soufflera
Je repartira
Dès que les vents tourneront
Nous nous en aillerons
\endchorus

\beginverse
C'est pas l'homme qui prend la mer
C'est la mer qui prend l'homme
Moi, la mer, elle m'a pris
Au dépourvu, tans pis
J'ai eu si mal au cœur
Sur la mer en furie
Que j'ai vomi mon quatre heures
Et mon minuit aussi
J' me suis cogné partout
J'ai dormi dans des draps mouillés
Ça m'a coûté ses sous
C'est de la plaisance, c'est le pied
\endverse

\beginverse
C'est pas l'homme qui prend la mer
C'est la mer qui prend l'homme
Mais elle prend pas la femme
Qui préfère la campagne
La mienne m'attend au port
Au bout de la jetée
L'horizon est bien mort
Dans ses yeux délavés
Assise sur une bitte
D'amarrage, elle pleure
Son homme qui la quitte
La mer c'est son malheur
\endverse

\beginverse
C'est pas l'homme qui prend la mer
C'est la mer qui prends l'homme
Moi, la mer, elle m'a pris
Comme on prend un taxi
Je ferai le tour du monde
Pour voir à chaque étape
Si tous les gars du monde
Veulent bien me lâcher la grappe
J'irais aux quatre vents
Foutre un peu le boxon
Jamais les océans
N'oublieront mon prénom
\endverse

\beginchorus
Dès que le vent soufflera
Je repartira
Dès que les vents tourneront
Nous nous en aillerons
Ho ho ho ho ho hissez haut ho ho ho
\endchorus

\beginverse
C'est pas l'homme qui prend la mer
C'est la mer qui prends l'homme
Moi, la mer, elle m'a pris
Et mon bateau aussi
Il est fier, mon navire
Il est beau, mon bateau
C'est un fameux trois mats
Fin comme un oiseau (Hissez haut)
Tabarly, Pageot
Kersauson ou Riguidel
Naviguent pas sur des cageots
Ni sur des poubelles
\endverse

\beginverse
C'est pas l'homme qui prend la mer
C'est la mer qui prends l'homme
Moi, la mer, elle m'a pris
Je me souviens un Vendredi
Ne pleure plus, ma mère
Ton fils est matelot
Ne pleure plus, mon père
Je vis au fil de l'eau
Regardez votre enfant
Il est parti marin
Je sais c'est pas marrant
Mais c'était mon destin
\endverse
\endsong


\beginsong{Les copains d'abord}[by={Brassens},sr={Obligatoire}]

\beginverse
Non, ce n'était pas le radeau
De la Méduse, ce bateau,
Qu'on se le dis' au fond des ports,
Dis' au fond des ports,
Il naviguait en pèr' peinard
Sur la grand-mare des canards,
Et s'app'lait les Copains d'abord
Les Copains d'abord.
\endverse

\beginverse
Ses fluctuat nec mergitur
C'était pas d'la littératur',
N'en déplaise aux jeteurs de sort,
Aux jeteurs de sort,
Son capitaine et ses mat'lots
N'étaient pas des enfants d'salauds,
Mais des amis franco de port,
Des copains d'abord.
\endverse

\beginverse
C'étaient pas des amis de lux',
Des petits Castor et Pollux,
Des gens de Sodome et Gomorrh',
Sodome et Gomorrh',
C'étaient pas des amis choisis
Par Montaigne et La Boeti',
Sur le ventre ils se tapaient fort,
Les copains d'abord.
\endverse

\beginverse
C'étaient pas des anges non plus,
L'Evangile, ils l'avaient pas lu,
Mais ils s'aimaient tout's voil's dehors,
Tout's voil's dehors,
Jean, Pierre, Paul et compagnie,
C'était leur seule litanie
Leur Crédo, leur Confitéor,
Aux copains d'abord.
\endverse

\beginverse
Au moindre coup de Trafalgar,
C'est l'amitié qui prenait l'quart,
C'est elle qui leur montrait le nord,
Leur montrait le nord.
Et quand ils étaient en détresse,
Qu'leur bras lancaient des S.O.S.,
On aurait dit les sémaphores,
Les copains d'abord.
\endverse

\beginverse
Au rendez-vous des bons copains,
Y'avait pas souvent de lapins,
Quand l'un d'entre eux manquait a bord,
C'est qu'il était mort.
Oui, mais jamais, au grand jamais,
Son trou dans l'eau n'se refermait,
Cent ans après, coquin de sort !
Il manquait encor.
\endverse

\beginverse
Des bateaux j'en ai pris beaucoup,
Mais le seul qui'ait tenu le coup,
Qui n'ait jamais viré de bord,
Mais viré de port,
Naviguait en père peinard
Sur la grand-mare des canards,
Et s'app'lait les Copains d'abord
Les Copains d'abord.
\endverse
\endsong


\beginsong{Ma môme}[by={Jean Ferrat},sr={}]

\beginverse
Ma môme, ell' joue pas les starlettes
Ell' met pas des lunettes
De soleil
Ell' pos' pas pour les magazines
Ell' travaille en usine
A Créteil
\endverse

\beginverse
Dans une banlieue surpeuplée
On habite un meublé
Elle et moi
La fenêtre n'a qu'un carreau
Qui donne sur l'entrepôt
Et les toits
\endverse

\beginverse
On va pas à Saint-Paul-de-Vence
On pass' tout's nos vacances
A Saint-Ouen
Comme famille on n'a qu'une marraine
Quelque part en Lorraine
Et c'est loin
\endverse

\beginverse
Mais ma môme elle a vingt-cinq berges
Et j'crois bien qu'la Saint'Vierge
Des églises
N'a pas plus d'amour dans les yeux
Et ne sourit pas mieux
Quoi qu'on dise
\endverse

\beginverse
L'été quand la vill' s'ensommeille
Chez nous y a du soleil
Qui s'attarde
Je pose ma tête sur ses reins
Je prends douc'ment sa main
Et j'la garde
\endverse

\beginverse
On s'dit toutes les choses qui nous viennent
C'est beau comm' du Verlaine
On dirait
On regarde tomber le jour
Et puis on fait l'amour
En secret
\endverse

\beginverse
Ma môme, ell' joue pas les starlettes
Ell' met pas des lunettes
De soleil
Ell' pos' pas pour les magazines
Ell' travaille en usine
A Créteil
\endverse
\endsong

\beginsong{San Francisco}[by={Maxime Le Forestier},sr={}]

\beginverse
C'est une maison bleue
Adossée à la colline
On y vient à pied
On ne frappe pas
Ceux qui vivent là
Ont jeté la clé
On se retrouve ensemble
Après des années de route
Et on vient s'asseoir
Autour du repas
Tout le monde est là
À cinq heures du soir
\endverse

\beginchorus
Quand San Francisco s'embrume
Quand San Francisco s'allume
San Francisco….
Où êtes-vous
Lizzard et Luc?
Psylvia, attendez- moi.
\endchorus

\beginverse
Nageant dans le brouillard
Enlacés roulant dans l'herbe
On écoutera Tom à la guitare
Phil à la kéna jusqu'à la nuit noire.
Un autre arrivera
Pour nous dire des nouvelles
D'un qui reviendra
Dans un ans ou deux
Puisqu'il est heureux
On s'endormira
\endverse

\beginverse
C'est une maison bleue
Accrochée à ma mémoire
On y vient à pied
On ne frappe pas
Ceux qui vivent là
Ont jeté la clé
Peuplée de cheveux longs
De grands lits et de musique
Peuplée de lumière
Et peuplée de fous
Elle sera dernière
À rester debout
\endverse

\beginchorus
Si San Francisco s'effondre
Si San Francisco s'effondre
San Francisco….
Où êtes-vous
Lizzard et Luc?
Psylvia, attendez-moi.
\endchorus
\endsong

\beginsong{Couleur café}[by={Gainsbourg},sr={}]

\beginverse
J'aime ta couleur café
Tes cheveux café
Ta gorge café
J'aime quand pourmoi tu danses
Alors j'entends murmurer
Tous tes bracelets
Jolis bracelets
A tes pieds ils se balancent
\endverse

\beginchorus
Couleur café
Que j'aime ta couleur café
\endchorus

\beginverse
C'est quand même fou l'effet
L'effet que ça fait
De te voir rouler
Ainsi des yeux et des hanches
Si tu fais comme le café
Rien qu'à m'énerver
Rien qu'à m'exciter
Ce soir la nuit sera blanche
\endverse

\beginverse
L'amour sans philosopher
C'est comme le café
Très vite passé
Mais que veux-tu que j'y fasse
On en a marre de café
Et c'est terminé
Pour tout oublier
On attend que ça se tasse
\endverse
\endsong

\beginsong{La vie ne vaut rien}[by={Alain Souchon},sr={}]

\beginverse
Il a tourné sa vie dans tous les sens
Pour savoir si ça avait un sens l'existence
Il a demandé leur avis à des tas de gens ravis
Ravis, de donner leur avis sur la vie
Il a traversé les vapeurs des derviches tourneurs
Des haschich fumeurs et il a dit
\endverse

\beginchorus
La vie ne vaut rien, rien, la vie ne vaut rien
Mais moi quand je tiens, tiens
Là dans mes mains éblouies
Les deux jolis petits seins de mon amie
Là je dis rien, rien, rien, rien ne vaut la vie
\endchorus

\beginverse
Il a vu l'espace qui passe
Entre la jet set les fastes, les palaces
Et puis les techniciens de surface
D'autres espèrent dans les clochers, les monastères
Voir le vieux sergent pépère mais ce n'est que Richard Gere
Il est entré comme un insecte sur site d'Internet
Voir les gens des sectes et il a dit
\endverse

\beginverse
Il a vu manque d'amour, manque d'argent
Comme la vie c'est détergeant
Et comme ça nettoie les gens
Il a joué jeux interdit pour des amis endormis, la nostalgie
Et il a dit
\endverse
\endsong

\beginsong{Sensualité}[by={Axelle Red},sr={}]

\beginverse
Jamais je n'aurais pensé
"Tant besoin de lui"
Je me sens si envoûtée
Que ma maman me dit, ralentis
Désir ou amour
Tu le sauras un jour
\endverse

\beginchorus
J'aime j'aime
Tes yeux, j'aime ton odeur
Tous tes gestes en douceur
Lentement dirigés
Sensualité
Ouh stop un instant
J'aimerais que ce moment
Fixe pour des tas d'années
Ta sensualité
\endchorus

\beginverse
Il parait qu'après quelques temps
La passion s'affaiblit
Pas toujours apparemment
Et maman m'avait dit, ralentis
Désir ou amour tu le sauras un jour
\endverse

\beginverse
Je te demande si simplement
Ne fais pas semblant
Je t'aimerai encore
Et encore
Désir ou amour tu le sauras un jour
\endverse
\endsong

\beginsong{L'encre de tes yeux}[by={Francis Cabrel},sr={}]

\beginverse
Puisqu'on ne vivra jamais tout les deux
Puisqu'on est fou, puisqu'on est seul
Puisqu'ils sont si nombreux
Même la morale parle pour eux
\endverse

\beginchorus
J'aimerais quand même te dire
Tout ce que j'ai pu écrire
Je l'ai puisé à l'encre de tes yeux
\endchorus

\beginverse
Je n'avais pas vu que tu portais des chaînes
A trop vouloir te regarder
J'en oubliais les miennes
On rêvait de Venise et de liberté
\endverse

\beginverse
Tu viendras longtemps marcher dans mes rêves
Tu viendras toujours du côté
Où le soleil se lève
Et si malgré ça j'arrive à t'oublier
J'aimerais quand même te dire
Tout ce que j'ai pu écrire
Aura longtemps le parfum des regrets
\endverse

\beginverse
Mais puisqu'on ne vivra jamais tous les deux
Puisqu'on est fou, puisqu'on est seul
Puisqu'ils sont si nombreux
Même la morale parle pour eux
\endverse
\endsong

\beginsong{Le Sud}[by={Nino Ferrer},sr={}]

\beginverse
C'est un endroit qui ressemble à la Louisiane
À l'Italie
Il y a du linge étendu sur la terrasse
Et c'est joli
\endverse

\beginchorus
On dirait le Sud
Le temps dure longtemps
Et la vie sûrement
Plus d'un million d'années
Et toujours en été
\endchorus

\beginverse
Y'a plein d'enfants qui se roulent sur la pelouse
Y'a plein de chiens
Y'a même un chat, une tortue, des poissons rouges
Il ne manque rien
\endverse

\beginverse
Un jour ou l’autre, il faudra qu'il y ait la guerre
On le sait bien
On n'aime pas ça, mais on ne sait pas quoi faire
On dit : "C'est le destin"
\endverse

\beginchorus
Tant pis pour le Sud
C'était pourtant bien
On aurait pu vivre
Plus d'un million d’années
Et toujours en été
\endchorus
\endsong

\beginsong{Une belle histoire}[by={Michel Fuguain},sr={}]
\endsong

\beginsong{La poupée qui fait non}[by={Michel Polnareff},sr={}]
\endsong

\beginsong{On ira tous au paradis}[by={Michel Polnareff},sr={}]
\endsong

\beginsong{Emmenez moi}[by={Charles Aznavour},sr={}]
\endsong

\beginsong{La bombe humaine}[by={Téléphone},sr={}]
\endsong

\beginsong{Temps à nouveau}[by={Jean-Louis Aubert},sr={}]
\endsong

\beginsong{Rock Collection}[by={Laurent Voulzy},sr={}]
\endsong

\beginsong{Morgane de toi}[by={Renaud},sr={}]
\endsong

\beginsong{La Mer}[by={Charles Trenet},sr={}]
\endsong

\beginsong{Laisse béton}[by={Renaud},sr={}]
\endsong

\beginsong{Il y a}[by={JJG},sr={}]
\endsong

\beginsong{Les Corons}[by={Bachelet},sr={}]
\endsong


\end{songs}
\end{document}
