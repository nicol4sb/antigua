\documentclass{article}
\usepackage[chorded]{songs}
\usepackage{biblatex}
\usepackage[landscape]{geometry}

%\noversenumbers

\begin{document}

\begin{titlepage}
   \vspace*{\stretch{1.0}}
   \begin{center}
      \Huge\textbf{The Antigua songbook}\\
      \Large\textit{Vaga, Sylvestre, Cruising Bird}
   \end{center}
   \vspace*{\stretch{2.0}}
\end{titlepage}
\songsection{Chanson française}

\begin{songs}{}
\linespread{0.1}

\beginsong{Il est libre Max}[by={Hervé Christiani},sr={Laurent à Antigua}]

\beginverse
\gtab{Em}{022000} \gtab{C}{032010}\gtab{D}{000232}
\[Em]Il met de la magie mine de rien \[C]dans tout c'qu'il fait
Il \[D]a le sourire facile même pour \[Em]les imbéciles
Il \[Em]s'amuse bien il tombe \[C]jamais dans les pièges
\[D]Il s'laisse pas étourdir par \[Em]les néons des manêges
Il \[Em]vit sa vie sans s'occuper \[C]des grimaces
Que font \[D]autour de lui les poissons\[Em] dans la nasse
\endverse

\beginchorus
\[Em]Il est libre Max, \[C]il est libre Max
\[D]Y'en a même qui disent qu'ils l'ont \[Em]vu voler
\endchorus

\beginverse
Il travaille un p'tit peu quand son orps est d'accord
Pour lui faut pas s'en faire il sait doser son effort
Dans l'panier d'crabes il joue pas les homards
Il cherche pas à tout prix à faire des bulles dans la mare
\endverse

\beginverse
Il r'garde autour de lui avec les yeux de l'amour
Avant qu't'aies rien pu dire il t'aime déjà au départ
Il fait pas d'bruit il joue pas du tambour
Mais la statue de marbre lui sourit dans la cour
\endverse

\beginverse
Et bien sûr toutes les filles lui font leurs yeux de velours
Lui pour leur faire plaisir il leur raconte des histoires
Il les emmène par delà les labours
Chevaucher les licornes à la tombée du soir
\endverse

\beginverse
Comme il n'a pas d'argent pour faire le grand voyageur
Il va parler souvent aux habitants de son coeur
Qu'est-ce qui s'racontent c'est ça qu'il faudrait savoir
Pour avoir comme lui autant d'amour dans l'regard
\endverse

\endsong

\beginsong{La cabane du pêcheur}[by={Francis Cabrel},sr={demandé par Sophie}]
\beginverse
\[E]Le soir \[B7/4]tombait de tout son \[E]poids \[B7/4]
\[E]Au dess\[B7/4]us de la ri\[E]vière \[B7/4]
\[C#m]Je rangeais \[B]mes cannes
On ne voyait plus \[A]que  \[E]du \[B7/4]feu
\[E]Je l'ai vu \[B7/4]s'approcher
\[E]La tête aille\[B7/4]urs dans ses \[E]prières \[B7/4]
\[C#m]Il m'a semblé \[B]voir trop briller \[A]ses yeux

Huuumm  -- Je lui  \[B7/4] ai dit     \[E]     \[B7/4]
E Si tu p \[B7/4] leures pour\[E] un garçon \[B7/4]
E Tu seras \[B7/4]  pas la der\[E]nière \[B7/4]
\[C#m] Souvent les poi B ssons sont bien pl A  us affectueux
E Va faire un petit t \[B7/4] our, respir\[E]e le grand a \[B7/4]  ir
\[C#m] Après je te parlerai de l'am B  our
Si je me souviens un peu  A
\endverse

\beginchorus
ouhhh\[E]Elle m' \[B7/4] a dit    \[E]      \[B7/4]
\[F#m] Elle m'a dit justement c'est ce que je voudra B  is savoir
Et j'ai dit " A  viens t'asseoir dans la \[C#m]  cabane du pécheur  D
C'est un mauva A is rêve, oublie-l\[E] e !
Tes rêves s \[F#m] ont toujours trop clair B  s ou trop noirs
Alors  A viens faire toi-même le mélange des couleurs   \[C#m]
Sur les murs de la cabane du pécheur   A
Viens t'asseoir " \[E]                \[B7/4]
Je lui ai dit
Le monde est pourtant pas si loin
On voit les lumières
Et la terre peut faire
Tous les bruits qu'elle veut
Y'a sûrement quelqu'un qui écoute
La-haut dans l'univers
Peut-être tu demandes plus qu'il ne peut ?
\endchorus
\endsong

\beginsong{Mon titre}[by={l'auteur},sr={demandé par Sophie}]

\beginverse
couplet
\endverse

\beginchorus
refrain
\endchorus

\endsong


\end{songs}
\end{document}
