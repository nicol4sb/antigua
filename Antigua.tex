\documentclass{article}
\usepackage[chorded]{songs}
\usepackage{biblatex}
\usepackage[landscape]{geometry}

%\noversenumbers

\newindex{titleidx}{index_des_chansons}
\newauthorindex{authidx}{index_des_auteurs}


\begin{document}

\begin{titlepage}
   \vspace*{\stretch{1.0}}
   \begin{center}
      \Huge\textbf{The Antigua songbook}\\
      \Large\textit{Vaga, Sylvestre, Mia, Cruising Bird}
   \end{center}
   \vspace*{\stretch{2.0}}
\end{titlepage}

\tableofcontents
\showindex{Index des chansons}{titleidx}
\showindex{Index des auteurs}{authidx}

\begin{songs}{}

%\songsection{Chanson française}

\beginsong{Il est libre Max}[by={Hervé Christiani},sr={Lolo solo},index={Il est libre Max}]]

\beginverse
\gtab{Em}{022000} \gtab{C}{032010}\gtab{D}{000232}
\[Em]Il met de la magie mine de rien \[C]dans tout c'qu'il fait
Il \[D]a le sourire facile même pour \[Em]les imbéciles
Il \[Em]s'amuse bien il tombe \[C]jamais dans les pièges
\[D]Il s'laisse pas étourdir par \[Em]les néons des manêges
Il \[Em]vit sa vie sans s'occuper \[C]des grimaces
Que font \[D]autour de lui les poissons\[Em] dans la nasse
\endverse

\beginchorus
\[Em]Il est libre Max, \[C]il est libre Max
\[D]Y'en a même qui disent qu'ils l'ont \[Em]vu voler
\endchorus

\beginverse
Il travaille un p'tit peu quand son orps est d'accord
Pour lui faut pas s'en faire il sait doser son effort
Dans l'panier d'crabes il joue pas les homards
Il cherche pas à tout prix à faire des bulles dans la mare
\endverse

\beginverse
Il r'garde autour de lui avec les yeux de l'amour
Avant qu't'aies rien pu dire il t'aime déjà au départ
Il fait pas d'bruit il joue pas du tambour
Mais la statue de marbre lui sourit dans la cour
\endverse

\beginverse
Et bien sûr toutes les filles lui font leurs yeux de velours
Lui pour leur faire plaisir il leur raconte des histoires
Il les emmène par delà les labours
Chevaucher les licornes à la tombée du soir
\endverse

\beginverse
Comme il n'a pas d'argent pour faire le grand voyageur
Il va parler souvent aux habitants de son coeur
Qu'est-ce qui s'racontent c'est ça qu'il faudrait savoir
Pour avoir comme lui autant d'amour dans l'regard
\endverse

\endsong

\beginsong{La cabane du pêcheur}[by={Francis Cabrel},sr={demandé par Sophie}]

\gtab{E}{022100} \gtab{B7/4}{0:022202} \gtab{C#m}{4:113321} \gtab{F#m}{2:133211}
\beginverse
\[E]Le soir \[B7/4]tombait de tout son \[E]poids \[B7/4]
\[E]Au dess\[B7/4]us de la ri\[E]vière \[B7/4]
\[C#m]Je rangeais \[B]mes cannes
On ne voyait plus \[A]que  \[E]du \[B7/4]feu
\[E]Je l'ai vu \[B7/4]s'approcher
\[E]La tête aille\[B7/4]urs dans ses \[E]prières \[B7/4]
\[C#m]Il m'a semblé \[B]voir trop briller \[A]ses yeux

Huuumm  -- Je lui  \[B7/4] ai dit  \[E] \[B7/4]
\[E]Si tu p\[B7/4]leures pour\[E] un garçon \[B7/4]
\[E]Tu seras \[B7/4]  pas la der\[E]nière \[B7/4]
\[C#m]Souvent les \[B]poissons sont bien pl\[A]  us affectueux
\[E] Va faire un petit t \[B7/4] our, respir\[E]e le grand a\[B7/4]ir
\[C#m]Après je te parlerai de l'am\[B]our
Si je me souviens un peu \[A]
\endverse

\beginchorus
ouhhh\[E]Elle m' \[B7/4] a dit    \[E]      \[B7/4]
\[F#m] Elle m'a dit justement c'est ce que je voudra\[B]is savoir
Et j'ai dit "\[A]  viens t'asseoir dans la \[C#m]  cabane du pécheur \[D]
C'est un mauva A is rêve, oublie-l\[E] e !
Tes rêves s \[F#m] ont toujours trop clair B  s ou trop noirs
Alors \[A] viens faire toi-même le mélange des couleurs   \[C#m]
Sur les murs de la cabane du pécheur  \[A]
Viens t'asseoir " \[E]                \[B7/4]
\endchorus
\beginverse
Je lui ai dit
Le monde est pourtant pas si loin
On voit les lumières
Et la terre peut faire
Tous les bruits qu'elle veut
Y'a sûrement quelqu'un qui écoute
La-haut dans l'univers
Peut-être tu demandes plus qu'il ne peut ?
\endverse
\endsong

\beginsong{Femme libérée}[by={Christian Dingler},sr={demandé par Bibi}]

\beginverse
Guitare : Am - F  - C - G x 1000 - Spéciale Lolo

Elle est abonnée à Marie-Claire
Dans l'Nouvel Ob's elle ne lit que Bretecher
Le Monde y'a longtemps qu'elle fait plus semblant
Elle achète Match en cachette c'est bien plus marrant
\endverse

\beginchorus
Ne la laisse pas tomber
Elle est si fragile
Etre une femme libérée tu sais c'est pas si facile
Ne la laisse pas tomber
Elle est si fragile
Etre une femme libérée tu sais c'est pas si facile
\endchorus

\beginverse
Au fond de son lit un macho s'endort
Qui ne l'aimera pas plus loin que l'aurore
Mais elle s'en fout elle s'éclate quand même
Et lui ronronne des tonnes de "Je t'aime"
\endverse

\beginverse
Sa première ride lui fait du souci
Le reflet du miroir pèse sur sa vie
Elle rentre son ventre à chaque fois qu'elle sort
Même dans "Elle" ils disent qu'il faut faire un effort
\endverse

\beginverse
Elle fume beaucoup elle a des avis sur tout
Elle aime raconter qu'elle sait changer une roue
Elle avoue son âge celui de ses enfants
Et goûte même un p'tit joint de temps en temps
\endverse

\endsong

\beginsong{L'autre Finistère}[by={Les Innocents},sr={demandé par Sophie}]

\gtab{Cm}{x35543}

\beginverse
Guitare : G - F - Em - Cm - G

\[G]Comprendrais tu ma belle qu'un jour
fatigué j'aille me bri\[F]ser la voix
une dernière fois à cent \[Em]vingt décibels
contre un grand châtaigner
d\[Cm]'amour pour \[G]toi

\[G]Trouveraistu cruel, que le doigt sur la bouche
Je t'e\[F]mmène, hors des villes
En un fort, une p\[Em]resqu'île
Oublier nos duels, nos escarmouches
\[Cm]Nos peurs imbé\[G]ciles

G - F - Em - Cm - G
On irait y attendre, la fin des combats
Jeter aux vers, aux vautours, tous nos plus beaux discours
Ces mots qu'on rêvait d'entendre, et qui n'existent pas
\[Cm]Y devenir\[G] sourds\[A] \[C] \[C7]
\endverse

\beginchorus
\[F]Il est un es\[Dm7]tuaire à nos \[E]fleuves de soupirs
où l'eau mê\[B7]le nos mystères et nos bel\[G]les différences
j'y \[F]apprendrai à me \[Dm7]taire et tes \[Em]larmes retenir
dans cet \[F]autre Finistère aux longues \[G]plages de silence
\endchorus

\beginverse
Trouverais-tu cruel
Que le doigt sur la bouche
Je t'emmène, hors des villes
En un fort, une presqu'île
Oublier nos duels
Nos escarmouches
Nos peurs imbéciles
On irait y attendre
La fin des combats
Jeter aux vers, aux vautours
Tous nos plus beaux discours
Ces mots qu'on rêvait d'entendre
Et qui n'existent pas
Y devenir sourds
\endverse

\beginverse
Bien sûr on se figure
Que le monde est mal fait
Que les jours nous abiment
Comme de la toile de Nîmes
Qu'entre nous, il y a des murs
Qui jamais ne fissurent
Que même l'air nous opprime
Et puis on s'imagine
Des choses et des choses
Que nos liens c'est l'argile
Des promesses faciles
Sans voir que sous la patine
Du temps, il y a des roses
Des jardins fertiles
\endverse

\beginverse
Car là-haut dans le ciel
Si un jour je m'en vais
Ce que je voudrais de nous
Emporter avant tout
C'est le sucre, et le miel
Et le peu que l'on sait
N'être qu'à nous
\endverse
\endsong

\beginsong{Elle a les yeux revolvers}[by={Marc Lavoine},sr={demandé par Vivi qui le trouve trop sexy}]

\beginverse
\[D]Un peu spé\[G]ciale, elle est céli\[A]bataire
Le visage \[F#m]pâle, les cheveux en ar\[Bm]rière
Et j'aime \[A]ça
\[D]Elle se dess\[G]ine sous des jupes \[A]fendues
Et je de\[F#m]vine des histoires \[Bm]défendues
C'est comme \[A]ça
\[G]Tellement si \[F#m]belle quand elle \[Bm]sort
\[G]Tellement si \[F#m]belle, je l'aime tellement \[Bm]si fort
\endverse

\beginchorus
\[G]Elle a les \[A]yeux revolver\[D], elle a le re\[G]gard qui \[A]tue
Elle a \[F#m]tiré la première\[G], m'a touché\[A], c'est foutu
\[G]Elle a les \[A]yeux revolver\[D], elle a le \[G]regard qui \[A]tue
Elle a \[F#m]tiré la pre\[G]mière, elle m'a touché\[A], c'est foutu
\endchorus

\beginverse
Un peu larguée, un peu seule sur la terre
Les mains tendues, les cheveux en arrière
Et j'aime ça
A faire l'amour sur des malentendus
On vit toujours des moments défendus
C'est comme ça
Tellement si femme quand elle mord
Tellement si femme, je l'aime tellement si fort
\endverse

\beginverse
Son corps s'achève sous des draps inconnus
Et moi je rêve de gestes défendus
C'est comme ça
Un peu spéciale, elle est célibataire
Le visage pâle, les cheveux en arrière
Et j'aime ça
Tellement si femme quand elle dort
Tellement si belle, je l'aime tellement si fort
\endverse

\endsong

\beginsong{Les Champs Elysées}[by={Joe Dassin},sr={spéciale dédicace à Oscar né à Paris XII}]

\beginverse
Je m'\[C]baladais sur l'\[E7]avenue le \[Am]cœur ouvert à \[C7]l'inconnu
J'av\[F]ais envie de \[C]dire bonjour à \[D7]n'importe \[G7]qui
N'impo\[C]rte qui et \[E7]ce fut toi, \[Am]je t'ai dit \[C7]n'importe quoi
Il su\[F]ffisait de t\[C]e parler, pour \[Dm]t'apprivoiser \[G7] \[C]
\endverse

\beginchorus
\[C]Aux \[E7]Champs-Elysées,\[Am] \[C7] \[F]aux Champs\[C]-Elysées\[D7] \[G7]
\[C]Au soleil, \[E7]sous la pluie, \[Am]à midi ou à \[C7]minuit
\[F]Il y a tout ce que vous \[C]voulez aux Champs-\[Dm]Elysées \[G7] \[C]
\endchorus

\beginverse
Tu m'as dit "J'ai rendez-vous dans un sous-sol avec des fous
Qui vivent la guitare à la main, du soir au matin"
Alors je t'ai accompagnée, on a chanté, on a dansé
Et l'on n'a même pas pensé à s'embrasser
\endverse

\beginverse
Hier soir deux inconnus et ce matin sur l'avenue
Deux amoureux tout étourdis par la longue nuit
Et de l'Étoile à la Concorde, un orchestre à mille cordes
Tous les oiseaux du point du jour chantent l'amour
\endverse
\endsong

\beginsong{Africa}[by={Rose Laurens},sr={Pour Vivi la reine du dancefloor}]

\beginverse
Je suis amoureuse d'une terre sauvage
Un sorcier vaudou m'a peint le visage
Son gris-gris me suit au son des tam-tams
Parfum de magie sur ma peau blanche de femme
\endverse

\beginchorus
Africa
J'ai envie de danser comme toi
De m'offrir à ta loi
Africa
De bouger à me faire mal de toi
Et d'obéir à ta voix
Africa
\endchorus

\beginverse
Je danse pied nus sous un soleil rouge
Les dieux à genoux ont le cœur qui bouge
Le feu de mon corps devient un rebelle
Le cri des gourous a déchiré le ciel
\endverse

\beginverse
Dangereuse et sensuelle, sous ta pluie sucrée
Panthère ou gazelle je me suis couchée
Au creux de tes griffes je suis revenue
A l'ombre des cases je ferai ma tribu
\endverse
\endsong

\beginsong{Ne partons pas fâchés}[by={Raphaël},sr={Et c'est pour Looooooorent}]

\beginverse
Bien sur qu'on a perdu la guerre, bien sur que je le reconnais
bien sur la vie nous mets le compte, bien la vie c'est une enclume
bien sur que j'aimerais bien te montrer qu'ailleurs on ferait pas que fuir
et bien sur j'ai pas les moyens et quand les poches sont vides alors allons rire
\endverse

\beginchorus
Ne partons pas fâchés, ça n'en vaut pas la peine
\endchorus

\beginverse
Bien sur que les montagnes sont belles, bien sur qu'il y a des vallées
Et les enfants sautent sur les mines, bien sur dans une autre vallée
Bien sur que les poissons ont froids à se traîner la dans la mer
Bien sur que j'ai encore en moi comme un veau avalé de travers
\endverse

\beginverse
Bien sur j'ai la ville dans le ventre, bien sur j'ai vendu ma moto
bien sur je te trouve très jolie, j'ai vraiment envie de te sauter
bien sur la vie nous fait offense bien sur la vie nous fait misère
on ira aussi vite que le vent, même si on a bien souvent ramper
\endverse


\beginverse
Bien sur que je te trouve très belle, bien sur je t'emmènerai à la mer
y'a rien d'autre a faire qu'à se saouler, attendre le jugement dernier
Transplanter la haut dans le Ciel, y parait que c'est pas pareil !
y parait que la vie n'es jamais aussi belle que dans tes rêves que dans tes rêves
\endverse

\beginverse
Et si l'on ne fait rien,
Ne partons pas fâchés, ça n'en vaut pas la peine
Y parait que les petits moineaux...
\endverse
\endsong

\beginsong{Belle Île en Mer}[by={Laurent Voulzy},sr={Pour les langoustes}]

\beginchorus
Belle-Ile-en-Mer
Marie-Galante
Saint-Vincent
Loin Singapour
Seymour Ceylan
Vous c'est l'eau c'est l'eau
Qui vous sépare
Et vous laisse à part
\endchorus

\beginverse
Moi des souvenirs d'enfance
En France
Violence
Manque d'indulgence
Par les différences que j'ai
Café
Léger
Au lait mélangé
Séparé petit enfant
Tout comme vous
Je connais ce sentiment
De solitude et d'isolement
\endverse

\beginverse
Comme laissé tout seul en mer
Corsaire
Sur terre
Un peu solitaire
L'amour je 1' voyais passer
Ohé Ohé
Je l' voyais passer
Séparé petit enfant
Tout comme vous
Je connais ce sentiment
De solitude et d'isolement
\endverse

\beginchorus
Karudea
Calédonie
Ouessant
Vierges des mers
Toutes seules
Tout 1' temps
Vous c'est l'eau c'est l'eau
Qui vous sépare
Et vous laisse à part
Oh oh...
\endchorus

\endsong

\beginsong{La Vie par procuration}[by={JJG},sr={Pour les filles}]

\beginverse
Elle met du vieux pain sur son balcon
Pour attirer les moineaux, les pigeons
Elle vit sa vie par procuration
Devant son poste de télévision
\endverse

\beginverse
Levée sans réveil
Avec le soleil
Sans bruit, sans angoisse
La journée se passe
Repasser, poussière
Y'a toujours à faire
Repas solitaires
En points de repère
\endverse

\beginverse
La maison si nette
Qu'elle en est supecte
Comme tous ces endroits
Où l'on ne vit pas
Les êtres ont cédé
Perdu la bagarre
Les choses ont gagné
C'est leur territoire
\endverse

\beginverse
Le temps qui nous casse
Ne la change pas
Les vivants se fanent
Mais les ombres, pas
Tout va, tout fonctionne
Sans but, sans pourquoi
D'hiver en automne
Ni fièvre, ni froid
\endverse

\beginverse
Elle met du vieux pain sur son balcon
Pour attirer les moineaux, les pigeons
Elle vit sa vie par procuration
Devant son poste de télévision
\endverse

\beginverse
Elle apprend dans la presse à scandale
La vie des autres qui s'étale
Mais finalement, de moins pire en banal
Elle finira par trouver ça normal
\endverse

\beginverse
Elle met du vieux pain sur son balcon
Pour attirer les moineaux, les pigeons
\endverse

\beginverse
Des crèmes et des bains
Qui font la peau douce
Mais ça fait bien loin
Que personne ne la touche
Des mois, des années
Sans personne à aimer
Et jour après jour
L'oubli de l'amour
\endverse

\beginverse
Ses rêves et désirs
Si sages et possibles
Sans cri, sans délire
Sans inadmissible
Sur dix ou vingt pages
De photos banales
Bilan sans mystère
D'années sans lumière
\endverse

\beginverse
Elle met du vieux pain sur son balcon
\endverse
\endsong


\beginsong{Papaoutai}[by={Stromae},sr={Allez Damien}]

\beginverse
Dites-moi d'où il vient
Enfin je saurais où je vais
Maman dit que lorsqu'on cherche bien
On finit toujours par trouver
Elle dit qu'il n'est jamais très loin
Qu'il part très souvent travailler
Maman dit "travailler c'est bien"
Bien mieux qu'être mal accompagné
Pas vrai ?
Où est ton papa ?
Dis-moi où est ton papa ?
Sans même devoir lui parler
Il sait ce qui ne va pas
Ah sacré papa
Dis-moi où es-tu caché ?
Ça doit, faire au moins mille fois que j'ai
Compté mes doigts
\endverse

\beginchorus
Où t'es, papa où t'es ?
Où t'es, papa où t'es ?
Où t'es, papa où t'es ?
Où, t'es où, t'es où, papa où t'es ?
\endchorus

\beginverse
Quoi, qu'on y croit ou pas
Y aura bien un jour où on y croira plus
Un jour ou l'autre on sera tous papa
Et d'un jour à l'autre on aura disparu
Serons-nous détestables ?
Serons-nous admirables ?
Des géniteurs ou des génies ?
Dites-nous qui donne naissance aux irresponsables ?
Ah dites-nous qui, tient,
Tout le monde sait comment on fait les bébés
Mais personne sait comment on fait des papas
Monsieur Je-sais-tout en aurait hérité, c'est ça
Faut l'sucer d'son pouce ou quoi ?
Dites-nous où c'est caché, ça doit
Faire au moins mille fois qu'on a, bouffé nos doigts
\endverse

\beginverse
Où est ton papa ?
Dis-moi où est ton papa ?
Sans même devoir lui parler
Il sait ce qui ne va pas
Ah sacré papa
Dis-moi où es-tu caché ?
Ça doit, faire au moins mille fois que j'ai
Compté mes doigts
\endverse
\endsong

\beginsong{Ma Rue}[by={Zebda},sr={La jeunesse de Nico}]

\beginverse
Dans cette rue y avait
Des espagnols qui n'osaient pas
Montrer qu'ils étaient de vieux réfugiés
Qu'avaient fui les cons et les rois
Dans cette rue y avait
Des français qu'avaient pas de chance ils ont écrit "Vive la France"
Au fronton de leur maison
Dans cette rue y avait
Des portugais fiers, Comme les geôliers de la misère
Quelques arbres fruitiers
Et la pudeur de la terre
\endverse

\beginchorus
C'était
Ma rue, ma famille
Les mamans qui s'égosillent
C'était : va jouer aux billes
C'était ma rue
C'était pas Manille
Non c'était pas les Antilles
Le marteau ou la faucille
C'était ma rue
Les glaces à la vanille
Et les petites qui frétillent
Qui n'étaient pas si gentilles
C'était ma rue
Bonjour les anguilles
Les condés qui nous quadrillent
Moi c'était pas ma Bastille
C'était ma rue
\endchorus

\beginverse
Dans cette rue y avait
L'Afrique et son mea-culpa
D'avoir un autre dieu je crois
Y z'ont trouvé des cons et des croix
Dans cette rue y avait
Tous les ouvriers de la terre
Y z'ont construit des pieds à terre
Qu'ils n'habiteront jamais
Dans cette rue y avait
Des caravanes comme
Des chariots de la colère
Qu'ont pas peur de l'hiver
De la fureur de la terre
\endverse

\beginverse
Dans cette rue je crois
Les enfants n'étaient pas de glace
Quand passait le camion de glace
On tirait des langues étrangères
On était dans les bois
On avait des arcs et des flèches
Quand d'autres avaient des cannes à pêche
Mais l'école elle en veut pas
Un jour on s'est fâchés
On a tout brûlé on a pas eu peur de l'enfer
Quand on s'est réveillés
Derrière des barreaux en fer
\endverse
\endsong

\beginsong{Déjeuner en paix}[by={Stephan Eicher},sr={So gooood}]

\beginverse
J'abandonne sur une chaise le journal du matin
Les nouvelles sont mauvaises d'où qu'elles viennent
J'attends qu'elle se réveille et qu'elle se lève enfin
Je souffle sur les braises pour qu'elles prennent
\endverse

\beginverse
Cette fois je ne lui annoncerai pas
La dernière hécatombe
Je garderai pour moi ce que m'inspire le monde
Elle m'a dit qu'elle voulait si je le permettais
Déjeuner en paix, déjeuner en paix
\endverse

\beginverse
Je vais à la fenêtre et le ciel ce matin
N'est ni rose ni honnête pour la peine
" Est-ce que tout va si mal ? Est-ce que rien ne va bien ?
L'homme est un animal " me dit-elle
\endverse

\beginverse
Elle prend son café en riant
Elle me regarde à peine
Plus rien ne la surprend sur la nature humaine
C'est pourquoi elle voudrait enfin si je le permets
Déjeuner en paix, déjeuner en paix
\endverse

\beginverse
Je regarde sur la chaise le journal du matin
Les nouvelles sont mauvaises d'où qu'elles viennent
" Crois-tu qu'il va neiger ? " me demande-t-elle soudain
" Me feras-tu un bébé pour Noël ? "
\endverse

\beginverse
Et elle prend son café en riant
Elle me regarde à peine
Plus rien ne la surprend sur la nature humaine
C'est pourquoi elle voudrait enfin si je le permets
Déjeuner en paix, déjeuner en paix
\endverse
\endsong


\beginsong{Dès que le vent soufflera}[by={Renaud},sr={We did it}]

\beginverse
C'est pas l'homme qui prend la mer
C'est la mer qui prend l'homme, tatatin
Moi, la mer, elle m'a pris
Je m'souviens un mardi
J'ai troqué mes santiags
Et mon cuir un peu zone
Contre une paire de docksides
Et un vieux ciré jaune
J'ai déserté les crasses
Qui me disaient "Sois prudent"
La mer, c'est dégueulasse
Les poissons baisent dedans
\endverse

\beginchorus
Dès que le vent soufflera
Je repartira
Dès que les vents tourneront
Nous nous en aillerons
\endchorus

\beginverse
C'est pas l'homme qui prend la mer
C'est la mer qui prend l'homme
Moi, la mer, elle m'a pris
Au dépourvu, tans pis
J'ai eu si mal au cœur
Sur la mer en furie
Que j'ai vomi mon quatre heures
Et mon minuit aussi
J' me suis cogné partout
J'ai dormi dans des draps mouillés
Ça m'a coûté ses sous
C'est de la plaisance, c'est le pied
\endverse

\beginverse
C'est pas l'homme qui prend la mer
C'est la mer qui prend l'homme
Mais elle prend pas la femme
Qui préfère la campagne
La mienne m'attend au port
Au bout de la jetée
L'horizon est bien mort
Dans ses yeux délavés
Assise sur une bitte
D'amarrage, elle pleure
Son homme qui la quitte
La mer c'est son malheur
\endverse

\beginverse
C'est pas l'homme qui prend la mer
C'est la mer qui prends l'homme
Moi, la mer, elle m'a pris
Comme on prend un taxi
Je ferai le tour du monde
Pour voir à chaque étape
Si tous les gars du monde
Veulent bien me lâcher la grappe
J'irais aux quatre vents
Foutre un peu le boxon
Jamais les océans
N'oublieront mon prénom
\endverse

\beginchorus
Dès que le vent soufflera
Je repartira
Dès que les vents tourneront
Nous nous en aillerons
Ho ho ho ho ho hissez haut ho ho ho
\endchorus

\beginverse
C'est pas l'homme qui prend la mer
C'est la mer qui prends l'homme
Moi, la mer, elle m'a pris
Et mon bateau aussi
Il est fier, mon navire
Il est beau, mon bateau
C'est un fameux trois mats
Fin comme un oiseau (Hissez haut)
Tabarly, Pageot
Kersauson ou Riguidel
Naviguent pas sur des cageots
Ni sur des poubelles
\endverse

\beginverse
C'est pas l'homme qui prend la mer
C'est la mer qui prends l'homme
Moi, la mer, elle m'a pris
Je me souviens un Vendredi
Ne pleure plus, ma mère
Ton fils est matelot
Ne pleure plus, mon père
Je vis au fil de l'eau
Regardez votre enfant
Il est parti marin
Je sais c'est pas marrant
Mais c'était mon destin
\endverse
\endsong


\beginsong{Les copains d'abord}[by={Brassens},sr={Obligatoire}]

\beginverse
Non, ce n'était pas le radeau
De la Méduse, ce bateau,
Qu'on se le dis' au fond des ports,
Dis' au fond des ports,
Il naviguait en pèr' peinard
Sur la grand-mare des canards,
Et s'app'lait les Copains d'abord
Les Copains d'abord.
\endverse

\beginverse
Ses fluctuat nec mergitur
C'était pas d'la littératur',
N'en déplaise aux jeteurs de sort,
Aux jeteurs de sort,
Son capitaine et ses mat'lots
N'étaient pas des enfants d'salauds,
Mais des amis franco de port,
Des copains d'abord.
\endverse

\beginverse
C'étaient pas des amis de lux',
Des petits Castor et Pollux,
Des gens de Sodome et Gomorrh',
Sodome et Gomorrh',
C'étaient pas des amis choisis
Par Montaigne et La Boeti',
Sur le ventre ils se tapaient fort,
Les copains d'abord.
\endverse

\beginverse
C'étaient pas des anges non plus,
L'Evangile, ils l'avaient pas lu,
Mais ils s'aimaient tout's voil's dehors,
Tout's voil's dehors,
Jean, Pierre, Paul et compagnie,
C'était leur seule litanie
Leur Crédo, leur Confitéor,
Aux copains d'abord.
\endverse

\beginverse
Au moindre coup de Trafalgar,
C'est l'amitié qui prenait l'quart,
C'est elle qui leur montrait le nord,
Leur montrait le nord.
Et quand ils étaient en détresse,
Qu'leur bras lancaient des S.O.S.,
On aurait dit les sémaphores,
Les copains d'abord.
\endverse

\beginverse
Au rendez-vous des bons copains,
Y'avait pas souvent de lapins,
Quand l'un d'entre eux manquait a bord,
C'est qu'il était mort.
Oui, mais jamais, au grand jamais,
Son trou dans l'eau n'se refermait,
Cent ans après, coquin de sort !
Il manquait encor.
\endverse

\beginverse
Des bateaux j'en ai pris beaucoup,
Mais le seul qui'ait tenu le coup,
Qui n'ait jamais viré de bord,
Mais viré de port,
Naviguait en père peinard
Sur la grand-mare des canards,
Et s'app'lait les Copains d'abord
Les Copains d'abord.
\endverse
\endsong


\beginsong{Ma môme}[by={Jean Ferrat},sr={}]

\beginverse
Ma môme, ell' joue pas les starlettes
Ell' met pas des lunettes
De soleil
Ell' pos' pas pour les magazines
Ell' travaille en usine
A Créteil
\endverse

\beginverse
Dans une banlieue surpeuplée
On habite un meublé
Elle et moi
La fenêtre n'a qu'un carreau
Qui donne sur l'entrepôt
Et les toits
\endverse

\beginverse
On va pas à Saint-Paul-de-Vence
On pass' tout's nos vacances
A Saint-Ouen
Comme famille on n'a qu'une marraine
Quelque part en Lorraine
Et c'est loin
\endverse

\beginverse
Mais ma môme elle a vingt-cinq berges
Et j'crois bien qu'la Saint'Vierge
Des églises
N'a pas plus d'amour dans les yeux
Et ne sourit pas mieux
Quoi qu'on dise
\endverse

\beginverse
L'été quand la vill' s'ensommeille
Chez nous y a du soleil
Qui s'attarde
Je pose ma tête sur ses reins
Je prends douc'ment sa main
Et j'la garde
\endverse

\beginverse
On s'dit toutes les choses qui nous viennent
C'est beau comm' du Verlaine
On dirait
On regarde tomber le jour
Et puis on fait l'amour
En secret
\endverse

\beginverse
Ma môme, ell' joue pas les starlettes
Ell' met pas des lunettes
De soleil
Ell' pos' pas pour les magazines
Ell' travaille en usine
A Créteil
\endverse
\endsong

\beginsong{San Francisco}[by={Maxime Le Forestier},sr={}]

\beginverse
C'est une maison bleue
Adossée à la colline
On y vient à pied
On ne frappe pas
Ceux qui vivent là
Ont jeté la clé
On se retrouve ensemble
Après des années de route
Et on vient s'asseoir
Autour du repas
Tout le monde est là
À cinq heures du soir
\endverse

\beginchorus
Quand San Francisco s'embrume
Quand San Francisco s'allume
San Francisco….
Où êtes-vous
Lizzard et Luc?
Psylvia, attendez- moi.
\endchorus

\beginverse
Nageant dans le brouillard
Enlacés roulant dans l'herbe
On écoutera Tom à la guitare
Phil à la kéna jusqu'à la nuit noire.
Un autre arrivera
Pour nous dire des nouvelles
D'un qui reviendra
Dans un ans ou deux
Puisqu'il est heureux
On s'endormira
\endverse

\beginverse
C'est une maison bleue
Accrochée à ma mémoire
On y vient à pied
On ne frappe pas
Ceux qui vivent là
Ont jeté la clé
Peuplée de cheveux longs
De grands lits et de musique
Peuplée de lumière
Et peuplée de fous
Elle sera dernière
À rester debout
\endverse

\beginchorus
Si San Francisco s'effondre
Si San Francisco s'effondre
San Francisco….
Où êtes-vous
Lizzard et Luc?
Psylvia, attendez-moi.
\endchorus
\endsong

\beginsong{Couleur café}[by={Gainsbourg},sr={}]

\beginverse
J'aime ta couleur café
Tes cheveux café
Ta gorge café
J'aime quand pourmoi tu danses
Alors j'entends murmurer
Tous tes bracelets
Jolis bracelets
A tes pieds ils se balancent
\endverse

\beginchorus
Couleur café
Que j'aime ta couleur café
\endchorus

\beginverse
C'est quand même fou l'effet
L'effet que ça fait
De te voir rouler
Ainsi des yeux et des hanches
Si tu fais comme le café
Rien qu'à m'énerver
Rien qu'à m'exciter
Ce soir la nuit sera blanche
\endverse

\beginverse
L'amour sans philosopher
C'est comme le café
Très vite passé
Mais que veux-tu que j'y fasse
On en a marre de café
Et c'est terminé
Pour tout oublier
On attend que ça se tasse
\endverse
\endsong

\beginsong{La vie ne vaut rien}[by={Alain Souchon},sr={}]

\beginverse
Il a tourné sa vie dans tous les sens
Pour savoir si ça avait un sens l'existence
Il a demandé leur avis à des tas de gens ravis
Ravis, de donner leur avis sur la vie
Il a traversé les vapeurs des derviches tourneurs
Des haschich fumeurs et il a dit
\endverse

\beginchorus
La vie ne vaut rien, rien, la vie ne vaut rien
Mais moi quand je tiens, tiens
Là dans mes mains éblouies
Les deux jolis petits seins de mon amie
Là je dis rien, rien, rien, rien ne vaut la vie
\endchorus

\beginverse
Il a vu l'espace qui passe
Entre la jet set les fastes, les palaces
Et puis les techniciens de surface
D'autres espèrent dans les clochers, les monastères
Voir le vieux sergent pépère mais ce n'est que Richard Gere
Il est entré comme un insecte sur site d'Internet
Voir les gens des sectes et il a dit
\endverse

\beginverse
Il a vu manque d'amour, manque d'argent
Comme la vie c'est détergeant
Et comme ça nettoie les gens
Il a joué jeux interdit pour des amis endormis, la nostalgie
Et il a dit
\endverse
\endsong

\beginsong{Sensualité}[by={Axelle Red},sr={}]

\beginverse
Jamais je n'aurais pensé
"Tant besoin de lui"
Je me sens si envoûtée
Que ma maman me dit, ralentis
Désir ou amour
Tu le sauras un jour
\endverse

\beginchorus
J'aime j'aime
Tes yeux, j'aime ton odeur
Tous tes gestes en douceur
Lentement dirigés
Sensualité
Ouh stop un instant
J'aimerais que ce moment
Fixe pour des tas d'années
Ta sensualité
\endchorus

\beginverse
Il parait qu'après quelques temps
La passion s'affaiblit
Pas toujours apparemment
Et maman m'avait dit, ralentis
Désir ou amour tu le sauras un jour
\endverse

\beginverse
Je te demande si simplement
Ne fais pas semblant
Je t'aimerai encore
Et encore
Désir ou amour tu le sauras un jour
\endverse
\endsong

\beginsong{L'encre de tes yeux}[by={Francis Cabrel},sr={}]

\beginverse
Puisqu'on ne vivra jamais tout les deux
Puisqu'on est fou, puisqu'on est seul
Puisqu'ils sont si nombreux
Même la morale parle pour eux
\endverse

\beginchorus
J'aimerais quand même te dire
Tout ce que j'ai pu écrire
Je l'ai puisé à l'encre de tes yeux
\endchorus

\beginverse
Je n'avais pas vu que tu portais des chaînes
A trop vouloir te regarder
J'en oubliais les miennes
On rêvait de Venise et de liberté
\endverse

\beginverse
Tu viendras longtemps marcher dans mes rêves
Tu viendras toujours du côté
Où le soleil se lève
Et si malgré ça j'arrive à t'oublier
J'aimerais quand même te dire
Tout ce que j'ai pu écrire
Aura longtemps le parfum des regrets
\endverse

\beginverse
Mais puisqu'on ne vivra jamais tous les deux
Puisqu'on est fou, puisqu'on est seul
Puisqu'ils sont si nombreux
Même la morale parle pour eux
\endverse
\endsong

\beginsong{Le Sud}[by={Nino Ferrer},sr={}]

\beginverse
C'est un endroit qui ressemble à la Louisiane
À l'Italie
Il y a du linge étendu sur la terrasse
Et c'est joli
\endverse

\beginchorus
On dirait le Sud
Le temps dure longtemps
Et la vie sûrement
Plus d'un million d'années
Et toujours en été
\endchorus

\beginverse
Y'a plein d'enfants qui se roulent sur la pelouse
Y'a plein de chiens
Y'a même un chat, une tortue, des poissons rouges
Il ne manque rien
\endverse

\beginverse
Un jour ou l’autre, il faudra qu'il y ait la guerre
On le sait bien
On n'aime pas ça, mais on ne sait pas quoi faire
On dit : "C'est le destin"
\endverse

\beginchorus
Tant pis pour le Sud
C'était pourtant bien
On aurait pu vivre
Plus d'un million d’années
Et toujours en été
\endchorus
\endsong

\beginsong{La corrida}[by={Francis Cabrel},sr={}]
\beginverse
Depuis le \[Dm]temps que je patiente dans cette c\[F]hambre noire
J'entends qu'on s'\[C]amuse et qu'on chante au bout du c\[Bb2]ouloir.
Quelqu'un a \[Dm]touché le verrou et j'ai pl\[F] ongé vers le grand jour
J'ai vu les\[F]C anfares, les barrières et les gens\[Bb2] autour. \[Dm]     \[F]    \[C]    \[Bb2]
\endverse
\beginverse
Dans les prem\[Dm]iers moments j'ai cru qu'il fallait \[F] seulement se défendre
Mais cette p\[C]lace est sans issue, je commence à \[Bb2] comprendre.
Il ont ref\[Dm]ermé derrière moi, ils ont eu p\[F] eur que je recule
Je vais bien\[F]C inir par l'avoir cette \[Bb2] danseuse ridicule.
\endverse
\beginchorus
\[Dm]     \[F]   Est-ce que ce monde est \[C]sérieux ?  \[Bb2]
\[Dm]     \[F]   Est-ce que ce monde est \[C]sérieux ?  \[Bb2]
\endchorus
\beginverse
Andalou\[Dm]sie je me souviens, les prairies \[F] bordées de cactus.
Je ne vais pas tr\[C]embler devant ce pantin, c\[Bb2] e minus !
Je vais l'attr\[Dm]aper lui et son chapeau, les faire tour\[F] ner comme un soleil
Ce soir la\[C]femme du torero dormir\[Bb2]a sur ses deux oreilles.
\endverse
\beginchorus
\[Dm]    \[F]  Est-ce que ce monde est \[C]sérieux ?  \[Bb2]
\[Dm]    \[F]  Est-ce que ce monde est s\[C]érieux ?   \[Bb2]
J'en ai pour C2 suivi des fantômes, presque touché leurs \[Dm]ballerines \[D4].    \[Dm]
Ils ont frap\[Bb2]pé fort dans mon coup pour que je A  m'incline.\[A4]     \[A]
Ils sortent d'\[Bb2]où ces acrobates C2 , avec leurs costume\[A4]s de papie \[D4]r ?   \[Dm]
Je n'ai ja\[Bb2]mais appris à me battre contre des  C2 poupées\[Bb2].
\[Dm] \[F] \[C] \[Bb2]
\endchorus
\beginverse
Sentir le \[Dm]sable sous ma tête, c'est fou comme \[F]ça peut faire du bien
J'ai prie p\[C]our que tout s'arrête, Andalousie, \[Bb2]je me souviens.
Je les entends \[Dm]rire comme je râle, je les vois danser \[F]comme je succombe
Je ne pensais pas q\[C]u'on puisse autant s'amuser au\[Bb2]tour d'une tombe.
\endverse
\beginchorus
\[Dm]    \[F]  Est-ce que ce monde est \[C]sérieux ?  \[Bb2]
\[Dm]    \[F]  Est-ce que ce monde est s\[C]érieux ?  \[Bb2]
\[Dm] \[F] \[C] \[Bb2] ...
\endchorus
\beginverse
Si si Hombre Hombre Baila baila Hay que bailar de nuevo
Y mataremos otros Otras vidas, otros toros
Y mataremos otros Venga venga bailar y mataremos otros Venga, venga a bailar...
\endverse
\endsong


\beginsong{Une belle histoire}[by={Michel Fuguain},sr={}]
\beginverse
C'est un beau roman, c'est une belle histoire
C'est une romance d'aujourd'hui
Il rentrait chez lui, là-haut vers le brouillard
Elle descendait dans le midi, le midi
Ils se sont trouvés au bord du chemin
Sur l'autoroute des vacances
C'était sans doute un jour de chance
Ils avaient le ciel à portée de main
Un cadeau de la providence
Alors pourquoi penser au lendemain
\endverse
\beginverse
Ils se sont cachés dans un grand champ de blé
Se laissant porter par les courants
Se sont racontés leur vies qui commençaient
Ils n'étaient encore que des enfants, des enfants
Qui s'étaient trouvés au bord du chemin
Sur l'autoroute des vacances
C'était sans doute un jour de chance
Qui cueillirent le ciel au creux de leurs mains
Comme on cueille la providence
Refusant de penser au lendemain
\endverse
\beginverse
C'est un beau roman, c'est une belle histoire
C'est une romance d'aujourd'hui
Il rentrait chez lui, là-haut vers le brouillard
Elle descendait dans le midi, le midi
Ils se sont quittés au bord du matin
Sur l'autoroute des vacances
C'était fini le jour de chance
Ils reprirent alors chacun leur chemin
Saluèrent la providence en se faisant un signe de la main
\endverse
\endsong

\beginsong{La poupée qui fait non}[by={Michel Polnareff},sr={}]
\beginchorus
C'est une poupée qui fait non...non...non...non...
Toute la journée elle fait non...non..non...non...
\endchorus
\beginverse
Elle est... elle est tell'ment jolie
Que j'en rêve la nuit.
\endverse
\beginchorus
C'est une poupée qui fait non...non...non...non...
Toute la journée, elle fait non...non...non..non...
\endchorus
\beginverse
Personne ne lui a jamais appris
Qu'on pouvait dire oui.
Non...non...non...non...
Non...non...non...non...
\endverse
\beginchorus
Sans même écouter, elle fait non...non...non...non...
Sans même regarder, elle fait non...non...non...non...
\endchorus
\beginverse
Pourtant je donnerais ma vie
Pourqu'elle dise oui. (bis)
\endverse
\beginchorus
Mais c'est une poupée qui fait non...non...non...non...
Toute la journée elle fait non...non...non...non...
\endchorus
\beginverse
Personne ne lui a jamais appris
Que l'on peut dire oui...
Non...non...non...non...
Non...non...non...non...
\endverse
\endsong

\beginsong{On ira tous au paradis}[by={Michel Polnareff},sr={}]
\beginverse
On ira tous au paradis mêm' moi
Qu'on soit béni ou qu'on soit maudit, on ira
Toutes les bonnes soeurs et tous les voleurs
Toutes les brebis et tous les bandits
On ira tous au paradis
On ira tous au paradis, mêm' moi
Qu'on soit béni ou qu'on soit maudit, on ira
Avec les saints et les assassins
Les femmes du monde et puis les putains
On ira tous au paradis
\endverse
\beginverse
Ne crois pas ce que les gens disent
C'est ton coeur qui est la seule église
Laisse un peu de vague à ton âme
N'aie pas peur de la couleur des flammes de l'enfer
\endverse
\beginverse
On ira tous au paradis, mêm' moi
Qu'on croie en Dieu ou qu'on n'y croie pas, on ira...
Qu'on ait fait le bien ou bien Ie mal
On sera tous invités au bal
On ira tous au paradis
On ira tous au paradis, mêm' moi
Qu'on croie en Dieu ou qu'on n'y croie pas, on ira
Avec les chrétiens, avec les païens
Et même les chiens et même les requins
On ira tous au paradis
\endverse
\beginverse
On ira tous au paradis, mêm' moi,
Qu'on soit béni ou qu'on soit maudit, on ira
Tout' les bonnes soeurs et tous les voleurs
Tout' les brebis et tous les bandits
On ira tous au paradis
On ira tous au paradis, mêm' moi
Qu'on soit béni ou qu'on soit maudit, on ira
Tout'
Et puis...
Et puis...
Et tous les...
On ira tous au paradis
\endverse
\beginverse
On ira tous au paradis, mêm' moi
Qu'on soit béni ou qu'on soit maudit, on ira
Tout' les bonnes soeurs et tous les voleurs
Tout' les brebis et tous les bandits
On ira tous au paradis...
Surtout moi
\endverse
\endsong

\beginsong{Emmenez moi}[by={Charles Aznavour},sr={}]
\beginverse
Vers les docks, où le poids et l'ennui
Me courbent le dos
Ils arrivent, le ventre alourdi de fruits,
Les bateaux
\endverse
\beginverse
Ils viennent du bout du monde
Apportant avec eux des idées vagabondes
Aux reflets de ciel bleu, de mirages
Traînant un parfum poivré
De pays inconnus
Et d'éternels étés,
Où l'on vit presque nu,
Sur les plages
\endverse
\beginverse
Moi qui n'ai connu, toute ma vie,
Que le ciel du nord
J'aimerais débarbouiller ce gris
En virant de bord
\endverse
\beginverse
Emmenez-moi au bout de la terre
Emmenez-moi au pays des merveilles
Il me semble que la misère
Serait moins pénible au soleil
\endverse
\beginverse
Dans les bars, à la tombée du jour,
Avec les marins
Quand on parle de filles et d'amour,
Un verre à la main
\endverse
\beginverse
Je perds la notion des choses
Et soudain ma pensée m'enlève et me dépose
Un merveilleux été, sur la grève
Où je vois, tendant les bras,
L'amour qui, comme un fou, court au devant de moi
Et je me pends au cou de mon rêve
\endverse
\beginverse
Quand les bars ferment, et que les marins
Rejoignent leurs bords
Moi je rêve encore jusqu'au matin,
Debout sur le port
\endverse
\beginverse
Emmenez-moi au bout de la terre
Emmenez-moi au pays des merveilles
Il me semble que la misère
Serait moins pénible au soleil
\endverse
\beginverse
Un beau jour, sur un raffiot craquant
De la coque au pont
Pour partir, je travaillerai dans
La soute à charbon
\endverse
\beginverse
Prenant la route qui mène
A mes rêves d'enfant, sur des îles lointaines,
Où rien n'est important que de vivre
Où les filles alanguies
Vous ravissent le coeur en tressant, m'a-t-on dit
De ces colliers de fleurs qui enivrent
\endverse
\beginverse
Je fuirai, laissant là mon passé,
Sans aucun remords
Sans bagage et le coeur libéré,
En chantant très fort
\endverse
\beginverse
Emmenez-moi au bout de la terre
Emmenez-moi au pays des merveilles
Il me semble que la misère
Serait moins pénible au soleil
\endverse
\beginverse
Emmenez-moi au bout de la terre
Emmenez-moi au pays des merveilles
Il me semble que la misère
Serait moins pénible au soleil
\endverse
\endsong

\beginsong{La bombe humaine}[by={Téléphone},sr={}]
\beginverse
Je veux vous parler de l'arme de demain
Enfantée du monde elle en sera la fin
Je veux vous parler de moi, de vous
Je vois a l'intérieur des images, des couleurs
Qui ne sont pas a moi qui parfois me font peur
Sensations qui peuvent me rendre fou
Nos sens sont nos fils nous pauvres marionnettes
Nos sens sont le chemin qui mène droit a nos têtes
\endverse
\beginverse
La bombe humaine tu la tiens dans ta main
Tu as l'détonateur juste à côté du cœur
La bombe humaine c'est toi elle t'appartient
Si tu laisses quelqu'un prendre en main ton destin
C'est la fin, hum la fin, hum la fin, hum la fin
\endverse
\beginverse
Mon père ne dort plus sans prendre ses calmants
Maman ne travaille plus sans ses excitants
Quelqu'un leur vend de quoi tenir le coup
Je suis un électron bombardé de protons
Le rythme de la ville c'est ça mon vrai patron
\endverse
\beginverse
Je suis chargé d'électricité
Si par malheur au cœur de l'accélérateur
J'rencontre une particule qui m'mette de sale humeur
Oh non, faudrait pas que j'me laisse aller
Faudrait pas que j'me laisse aller
Faudrait pas que j'me laisse aller
Faudrait pas que j'me laisse aller
Faudrait pas que j'me laisse aller
Faudrait pas que j'me laisse aller
Faudrait pas que j'me laisse aller
Faudrait pas que j'me laisse aller
\endverse
\beginverse
La bombe humaine c'est l'arme de demain
Enfantée du monde elle en sera la fin
La bombe humaine c'est toi elle t'appartient
Si tu laisses quelqu'un prendre en main ton destin
C'est la fin
\endverse
\beginverse
La bombe humaine, tu la tiens dans ta main
Tu as l'détonateur juste à côtée du cœur
La bombe humaine, c'est toi elle t'appartient
Si tu laisses quelqu'un prendre ce qui te tient
C'est...
\endverse
\beginverse
La bombe humaine, tu la tiens dans ta main
Tu as l'détonateur juste a cote du cœur
La bombe humaine, c'est toi elle t'appartient
Si tu laisses quelqu'un prendre en main ton destin
C'est...
\endverse
\beginverse
La bombe humaine c'est l'arme de demain
La bombe humaine c'est toi elle t'appartient
La bombe humaine, tu la tiens dans ta main
Si tu laisses quelqu'un prendre ce qui te tient
C'est...
\endverse
\beginverse
La bombe humaine c'est l'arme de demain
La bombe humaine c'est toi elle t'appartient
La bombe humaine, tu la tiens dans ta main
Si tu laisses quelqu'un prendre ce qui te tient
C'est la...
\endverse
\endsong

\beginsong{Temps à nouveau}[by={Jean-Louis Aubert},sr={}]
\beginverse
Puisque les dauphins sont des rois
Que seul le silence s'impose
Puisqu'il revient à qui de droit
De tenter les métamorphoses
Puisque les révolutions,
Se font maintenant à la maison
Et que lorsque le monde implose
Ce n'est qu'une nouvelle émission
Émission
\endverse
\beginverse
Il est temps à nouveau
Oh temps à nouveau
De prendre le souffle à nouveau
Il est temps à nouveau
Oh temps à nouveau
De nous jeter à l'eau
\endverse
\beginverse
Puisque ce n'est plus qu'un système
Et sa police américaine
De monde meilleur on ne parle plus
Tout juste sauver celui-là, celui-là
\endverse
\beginverse
Eh ! Il est temps à nouveau
Oh temps à nouveau
De prendre le souffle à nouveau
Il est temps à nouveau
Oh temps à nouveau
De nous jeter à l'eau
De nous jeter à l'eau
\endverse
\beginverse
Puisque je suis mon aquarium
Moi le poison, moi le poisson
Changé en homme
\endverse
\beginverse
Il est temps à nouveau
Oh temps à nouveau
De prendre le souffle à nouveau
Il est temps à nouveau
Oh temps à nouveau
De me jeter à l'eau
Oh temps à nouveau
Oh temps à nouveau
De prendre le souffle à nouveau
Il est temps à nouveau
Oh temps à nouveau
De nous jeter à l'eau
\endverse
\beginverse
Beau temps pour se jeter à l'eau
Beau temps pour se jeter à l'eau
Beau temps pour se jeter à l'eau
Beau temps pour se jeter à l'eau
\endverse
\beginverse
Oh temps à nouveau
Temps à nouveau
Temps à nouveau
Oh ! Temps à nouveau
De nous jeter à l'eau
Oh temps à nouveau
De nous jeter à l'eau
Oh temps à nouveau
De nous jeter à l'eau
\endverse
\endsong

\beginsong{Rock Collection}[by={Laurent Voulzy},sr={}]
\beginverse
On a tous dans l'coeur une petite fille oubliée
Une jupe plissée queue d'cheval à la sortie du lycée
On a tous dans l'coeur un morceau de fer à user
Un vieux scooter de reve pour faire le cirque dans le
quartier
Et la p'tite fille chantait
Et la p'tite fille chantait
Un truc qui m'colle encore au coeur et au corps
[Locomotion]
"Everybody's doing a brand new dance now
C'm'on baby do the locomotion..."
\endverse
\beginverse
On a tous dans l'coeur le ticket pour Liverpool
Sortie de scène hélicoptère pour échapper
à la foule
Excuse-me Sir mais j'entends plus Big Ben qui sonne
Les scarabées bourdonnent c'est la folie à London
Et les Beatles chantaient
Et les Beatles chantaient
Un truc qui m'colle'encore au coeur et au corps
[Hard day's night]
"It's been a hard day's night
And I've been workin' like a dog
It's been a hard day's night, yeah, yeah, yeah
Yeah..."
\endverse
\beginverse
A quoi ca va me servir d'aller m'faire couper les tifs
Est-ce que ma vie sera mieux une fois qu'j'aurais mon
certif'
Betty a rigolé devant ma boule à zéro
Je lui dis si ca te plaît pas
T'as qu'à te plaindre au dirlo
Et je me suis fait virer
Et les Beach Boys chantaient
Un truc qui m'colle encore au coeur et au corps
[Get around]
"Round round get around, I get around
Yeah, get around, ooohh, round round I get around
Get around, get around, get around..."
\endverse
\beginverse
On a tous dans l'coeur des vacances à Saint-Malo
Et des parents en maillot qui dansent chez Luis Mariano
Au " Camping des flots bleus ", je me traîne des
tonnes de cafard
Si j'avais bossé un peu je me serais payé une guitare
Et Saint-Malo dormait
Et les radios chantaient
Un truc qui m'colle encore au coeur et au corps
[Gloria]
"Gloria G L O R I A Gloria G L O R I A Gloria Gloria..."
\endverse
\beginverse
Au café de ma banlieue t'as vu la bande à Jimmy
Ça frime pas mal, ca roule autour du baby
Le pauvre Jimmy s'est fait piquer chez le disquaire, c'est
dingue
Avec un single des Stones caché sous ses fringues
Et les loulous roulaient
Et les cailloux chantaient
Un truc qui m'colle encore au coeur et au corps
[Satisfaction]
"I can't get no, I can't get no satisfaction, hey hey hey!"
\endverse
\beginverse
Le jour où je vais partir je sens bien que ca va faire
mal
Ma mère n'aime pas mon blouson et les franges de mon futal
Le long des autoroutes il y a de beaux paysages
J'ai ma guitare dans le dos et pas de rond pour le voyage
Et Bob Dylan chantait
Et Bob Dylan chantait
Un truc qui m'colle encore au coeur et au corps
[Mister Tambourine man]
\endverse
\beginverse
Laissez-moi passer j'ai mes papiers mon visa
Je suis déjà dans l'avion going to America
Meme si je reste ici que je passe ma vie à Nogent
J'aurai une vieille Chevrolet et dix huit filles dedans
Et les Bee Gees chantaient
Et les Bee Gees chantaient
[Massachusetts]
\endverse
\beginverse
Au printemps 66 je suis tombé fou amoureux
Ça m'a fait plutot mal j'avais de l'eau dans les yeux
Ma p' tite poupée je t'emmène dans le pays de mes
langueurs
Elle fait douceur douceur la musique que j'ai dans le coeur
Toute la nuit on s'aimait
Quand Donovan chantait
Un truc qui m'colle encore au coeur et au corps
[Mellow Yellow]
\endverse
\beginverse
Maintenant j'ai une guitare et je voyage organisé
Je me lève tous les jours trop tard
Et je vis aux Champs-Elysées
Je suis parti je ne sais où mais pas où je voulais aller
Dans ma tete y a des trous je me souviens plus des couplets
Y a des reves qui sont cassés
Des airs qui partent en fumée
Des trucs qui m'colle encore au coeur et au corps
[California dreaming]
\endverse
\endsong

\beginsong{Morgane de toi}[by={Renaud},sr={}]
\beginverse
Y'a un mariole, qu'y'a au moins quatre ans
Y veut t'piquer ta pelle et ton seau
Ta couche-culotte avec les bombecs dedans
Lolita défends-toi, fous-y un coup d'rateau dans l'dos
\endverse
\beginverse
Attends un peu avant d'te faire emmerder
Parces p'tits machos qui pensent qu'à une chose
Jouer au docteur non conventionné
J'y ai joué aussi je sais de quoi j'cause
\endverse
\beginverse
J'les connais bien les playboys des bacs à sable
J'draguais leurs mères avant d'connaître la tienne
Si tu les écoutes y t'f'ront porter leurs cartables
'reus'ment qu'j'suis là que j'te r'garde et que j't'aime
\endverse
\beginchorus
Lola
J'suis qu'un fantôme quand tu vas où j'suis pas
Tu sais ma môme que j'suis morgane de toi
\endchorus
\beginverse
Comme j'en ai marre de m'faire tatouer des machins
Qui m'font comme une bande dessinée sur la peau
J'ai écrit ton nom avec des clous dorés un par un
Plantés dans le cuir de mon blouson dans l'dos
\endverse
\beginverse
T'es la seule gonzesse que j'peux t'nir dans mes bras
Sans m'démettre une épaule sans plier sous ton poids
Tu pèses moins lourd qu'un moineau qui mange pas
Déploie jamais tes ailes, Lolita t'envole pas
\endverse
\beginverse
Avec tes miches de rat qu'on dirait des noisettes
Et ta peau plus sucrée qu'un pain au chocolat
Tu risques d'donner faim à un tas de p'tits mecs
Quand t'iras à l'école si jamais t'y vas
\endverse
\beginverse
Qu'ess-tu m'racontes ? Tu veux un petit frangin ?
Tu veux qu'j't'achète un ami pierrot ?
Eh ! Les bébés ça s'trouve pas dans les magasins et j'crois pas
Que ta mère voudra qu'j'lui fasse un petit dans l'dos
\endverse
\beginverse
Ben quoi Lola on est pas bien ensemble ?
Tu crois pas qu'on est déjà bien assez nombreux ?
T'entends pas ce bruit c'est le monde qui tremble
Sous les cris des enfants qui sont malheureux
\endverse
\beginverse
Allez viens avec moi j't'embarque dans ma galère
Dans mon arche y'a d'la place pour tous les marmots
Avant qu'ce monde devienne un grand cimetière
Faut profiter un peu du vent qu'on a dans l'dos
\endverse
\endsong

\beginsong{La Mer}[by={Charles Trenet},sr={}]
\beginverse
La mer
qu'on voit danser le long des golfes clairs
a des reflets d'argent,
la mer,
des reflets changeants
sous la pluie.
\endverse
\beginverse
La mer
au ciel d'été confond
ses blancs moutons
avec les anges si purs,
la mer bergère d'azur
infinie.
\endverse
\beginverse
Voyez,
près des étangs,
ces grands roseaux mouillés.
Voyez,
ces oiseaux blancs
et ces maisons rouillées.
\endverse
\beginverse
La mer
les a bercés
le long des golfes clairs
et d'une chanson d'amour,
la mer
a bercé mon cœur pour la vie
\endverse
\endsong

\beginsong{Laisse béton}[by={Renaud},sr={}]
\beginverse
J’étais tranquille, j'étais peinard
accoudé au flipper,
le type est entré dans le bar,
a commandé un jambon-beurre,
puis il s'est approché de moi,
pi y m'a regardé comme ça :
T'as des bottes, mon pote, elles me bottent !
j'parie qu'c'est des santiags,
viens faire un tour dans l'terrain vague,
j'vais t'apprendre un jeu rigolo
à grands coups de chaine de vélo
j'te fais tes bottes à la baston !
moi j'y ai dit :
Laisse béton !
\endverse
\beginchorus
Y m'a filé un beigne, j'y ai filé une torgnole,
m'a filé une châtaigne, j'lui ai filé mes grolles.
\endchorus
\beginverse
j'étais tranquille, j'étais peinard.
accoudé au comptoir,
le type est entré dans le bar,
a commandé un café noir,
puis il m'a tapé sur l'épaule
et m'a regardé d'un air drôle :
T'as un blouson, mecton l'est pas bidon !
moi j'me les gèle sur mon scooter,
avec ça j's'rai un vrai rocker,
viens faire un tour dans la ruelle.
j'te montrerai mon Opinel,
et j'te chourav'rai ton blouson ! Moi j'y ai dit :
Laisse béton !
\endverse
\beginchorus
Y m'a filé une beigne, j'y ai filé un marron,
m'a filé une châtaigne, j'y ai filé mon blouson.
\endchorus
\beginverse
J’étais tranquille, j'étais peinard,
je réparais ma mobylette,
le type a surgi sur l'boul'vard
sur sa grosse moto super-chouette,
s'est arrêté l'long du trottoir
et m'a regardé d'un air bête :
T'as l'même blue-jean que James Dean,
t'arrête ta frime !
j'parie qu'c'est un vrai Lévi Strauss,
il est carrément pas craignoss,
viens faire un tour derrière l'église,
histoire que je te dévalise
à grands coups de ceinturon ! Moi j'y ai dit :
Laisse béton !
\endverse
\beginchorus
Y m'a filé une beigne, j'ai filé une mandale,
m'a filé une châtaigne, j'y ai filé mon futal.
\endchorus
\beginverse
La morale de c'te pauvre histoire,
c'est qu'quand t'es tranquille et peinard
faut pas trop traîner dans les bars,
à moins d'être fringué en costard.
Quand à la fin d'une chanson,
tu t'retrouves à poil sans tes bottes.
faut avoir d'l'imagination
pour trouver une chute rigolote.
\endverse
\endsong

\beginsong{Il y a}[by={JJG},sr={}]
\beginverse
Il y a
Du thym de la bruyère
Et des bois de pin
Rien de bien malin
Il y a
Des ruisseaux, des clairières
Pas de quoi en faire
Un plat de ce coin
\endverse
\beginverse
Il y a
Des odeurs de menthe
Et des cheminées
Et des feux dedans
Il y a
Des jours et des nuits lentes
Et l'histoire absente
Banalement
\endverse
\beginchorus
Et loin de tout, loin de moi
C'est là que tu te sens chez toi
De là que tu pars, ou tu reviens chaque fois
Et où tout finira
\endchorus
\beginverse
Il y a
Des enfants, des grands-mères
Une petite église
Et un grand café
Il y a
Au fond du cimetière
Des joies, des misères
Et du temps passé
Il y a
Une petite école
Et des bancs de bois
Tout comme autrefois
Il y a
Des images qui collent
Au bout de tes doigts
Et ton coeur qui bat
\endverse
\beginchorus
Et loin de tout, loin de moi
C'est là que tu te sens chez toi
De là que tu pars, où tu reviens chaque fois
Et où tout finira
\endchorus
\beginverse
Et plus la terre est aride, et plus cet amour est grand
Comme un mineur à sa mine, un marin à son océan
Plus la nature est ingrate, avide de sueur et de boue
Parce qu'on a tant besoin que l'on ait besoin de nous
Elle porte les stigmates de leur peine et de leur sang
Comme un mère préfère un peu son plus fragile enfant
\endverse
\beginchorus
Et loin de tout, loin de moi
C'est là que tu te sens chez toi
De là que tu pars, où tu reviens chaque fois
Et où tout finira
\endchorus
\endsong

\beginsong{Les Corons}[by={Pierre Bachelet},sr={}]
\beginchorus
Au nord, c'étaient les corons
La terre c'était le charbon
Le ciel c'était l'horizon
Les hommes des mineurs de fond
\endchorus
\beginverse
Nos fenêtres donnaient sur des f'nêtres semblables
Et la pluie mouillait mon cartable
Mais mon père en rentrant avait les yeux si bleus
Que je croyais voir le ciel bleu
J'apprenais mes leçons, la joue contre son bras
Je crois qu'il était fier de moi
Il était généreux comme ceux du pays
Et je lui dois ce que je suis
\endverse
\beginchorus
Au nord, c'étaient les corons
La terre c'était le charbon
Le ciel c'était l'horizon
Les hommes des mineurs de fond
\endchorus
\beginverse
Et c'était mon enfance, et elle était heureuse
Dans la buée des lessiveuses
Et j'avais des terrils à défaut de montagnes
D'en haut je voyais la campagne
Mon père était "gueule noire" comme l'étaient ses parents
Ma mère avait les cheveux blancs
Ils étaient de la fosse, comme on est d'un pays
Grâce à eux je sais qui je suis
\endverse
\beginchorus
Au nord, c'étaient les corons
La terre c'était le charbon
Le ciel c'était l'horizon
Les hommes des mineurs de fond
\endchorus
\beginverse
Y avait à la mairie le jour de la kermesse
Une photo de Jean Jaures
Et chaque verre de vin était un diamant rose
Posé sur fond de silicose
Ils parlaient de 36 et des coups de grisou
Des accidents du fond du trou
Ils aimaient leur métier comme on aime un pays
C'est avec eux que j'ai compris
\endverse
\beginchorus
Au nord, c'étaient les corons
La terre c'était le charbon
Le ciel c'était l'horizon
Les hommes des mineurs de fond
\endchorus
\endsong

\beginsong{Mélissa}[by={Julien Clerc},sr={}]
\beginverse
Mélissa métisse d'Ibiza vit toujours dévétue
Dites jamais que je vous ai dit ça ou Mélissa me tue
Le matin derrière ses canisses A-Lors qu'elle est moitié nue
Sur les murs devant chez Mélissa y'a tout plein d'inconnus
"Descendez ça c'est défendu! Oh c'est indécent!"
Elle crie mais bien entendu personne ne descend
\endverse
\beginverse
Sous la soie de sa jupe fendue en zoom en gros plan
Tout un tas d'individus fiment noirs et blancs
Mélissa, métisse d'Ibiza a des seins tous pointus
Dites jamais que je vous ai dit ça ou Mélissa me tue
"Descendez ça c'est défendu! Mater chez les gens"
Elle crie mais bien entendu y'a jamais d'agent
\endverse
\beginverse
Elle crie c'est du temps perdu personne ne l'entend
La police c'est tout des vendus dix ans qu'elle attend
Mélissa, métisse d'Ibiza a toujours sa vertu
Dites jamais que je vous ai dit ça ou Mélissa me tue
\endverse
\beginverse
Ouh! Matez ma métisse, ouh! Ma métisse est nue
Ouh! Matez ma métisse, ouh! Ma métisse est nue
Mélissa métisse d'Ibiza vit toujours dévétue
Dites jamais que je vous ai dit ça je vous ai jamais vu
Le matin derrière ses canisses A-Lors je vends des longues
vues
Mais si jamais Mélissa sait ça là c'est moi qui vous tue
\endverse
\beginverse
Ouh! Matez ma métisse, ouh! Ma métisse est nue
Ouh! Matez ma métisse, ouh! Ma métisse est nue...
\endverse
\endsong

\beginsong{Remember Paris}[by={Benabar},sr={}]

\gtab{C}{032010} 
\gtab{C7}{032310}
\gtab{F}{1:022100}
\gtab{Fm}{133111}
\gtab{Fm13}{131131}
\gtab{D7}{000212}
\gtab{G7}{320001}
\gtab{G13}{323000}
\newline

\beginverse
\[C]Do you remember Paris?\[Em]
\[C7]We met on Ile-Saint-Louis\[F]
\[Fm]I told you about the monuments\[C]
\[D7]You made fun of my accent\[G7]
\endverse

\beginverse
\[C]Do you remember Paris?\[Em]
\[C7]I wasn't good in history\[F]
\[Fm]To talk with you, to be useful\[C]
\[D7]Yes, I invented a little\[G7]
\endverse

\beginverse
\[F]I'm not sure that Paul Verlaine\[Fm13]
\[C]Used to live by the Seine\[A7]
\[D7]And I still don't know really\[G7]
\[G13]What is the Conciergerie\[G7]
\endverse

\beginverse
Do you remember Paris?
I told you "vous êtes jolie"
You had a city guide in hand
I had a few ideas in mind
\endverse

\beginverse
Do you remember Paris?
A restaurant place Clichy
I chose the wine pretending
To know what I was doing
\endverse

\beginverse
Do you remember me?
The next mornig, Roissy!
We kissed goodbye, you took your plane
I never see you again
\endverse

\beginverse
Do you remember Paris?
Do you remember me?
Mon accent est toujours là
Et moi? Moi, je me souviens de toi
Quand je m'balade au bord de la Seine
Je pense à une Américaine!
\endverse

\endsong


\end{songs}


\end{document}
